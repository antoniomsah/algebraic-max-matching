\section{Maximum matching problem}
Now, the maximum matching problem can be described as:
\\
\begin{problem}{MaxMatching}
	\label{prob:maxmatching}
	Given a graph \(G \coloneqq (V, E, \varphi)\) find a maximum matching of \(G\).
\end{problem}

\begin{definition}[Alternating and augmenting paths]
	Given a matching \(M\) of a graph \(G\).
	A path \(P\) is \(M\)-\textbf{alternating} if its edges are alternating in and out of \(M\). 
	Formally,
	\[
		e_i \in M \Longleftrightarrow e_{i+1} \not \in M \text { for each } i \in [l-1]
		\footnote{For \(n \in \Naturals\), we denote the set \(\{1, \dots, n\}\) as \([n]\).}
	\]
	And \(P\) is \(M\)-\textbf{augmenting} if both \(v_0\) and \(v_l\) are \(M\)-exposed.
\end{definition}

\begin{theorem}[Berge's theorem]
	Let \(G \coloneqq (V, E, \varphi)\) be a graph.
	A matching \(M\) is maximum if and only if there are no \(M\)-augmenting path.
\end{theorem}

\begin{proof}

\end{proof}

\begin{definition}[Vertex cover]
	For a graph \(G \coloneqq (V, E, \varphi)\), a subset \(K \subseteq V(G)\) is a \textbf{vertex cover} of \(G\) if every edge of \(E(G)\) has an end in in \(K\). 
	A vertex cover is said to be \textbf{minimal} if one has \(|K| \leq |K'|\) for every vertex cover \(K'\) of \(G\).
\end{definition}

\begin{theorem}[K\"onig's matching theorem]
	Let \(G\) be a bipartite graph, then the maximum size of a matching of \(G\) is equal to the minimum size of a vertex cover of \(G\).
\end{theorem}

\begin{proof}

\end{proof}

Note that K\"onig's theorem \textbf{does not} hold for all graphs;
It suffices to consider a single odd cycle.
