\section{Maximum matching problem}
Now, the maximum matching problem can be described as:
\\
\begin{problem}{MaxMatching}
	\label{prob:maxmatching}
	Given a graph \(G \coloneqq (V, E, \varphi)\) find a maximum matching of \(G\).
\end{problem}

\begin{definition}[Alternating and augmenting paths]
	\label{def:alt_path}
	Given a matching \(M\) of a graph \(G\).
	A path \(P\) is \(M\)-\textbf{alternating} if its edges are alternating in and out of \(M\). 
	Formally,
	\[
		e_i \in M \Longleftrightarrow e_{i+1} \not \in M \text { for each } i \in [l-1]
		\footnote{For \(n \in \Naturals\), we denote the set \(\{1, \dots, n\}\) as \([n]\).}
	\]
	And \(P\) is \(M\)-\textbf{augmenting} if both \(v_0\) and \(v_l\) are \(M\)-exposed.
\end{definition}

\begin{theorem}[Berge's theorem]
	\label{thm:berge}
	Let \(G \coloneqq (V, E, \varphi)\) be a graph.
	A matching \(M\) is maximum if and only if there are no \(M\)-augmenting path.
\end{theorem}

\begin{proof}
	Let \(G \coloneqq (V, E, \varphi)\) be a graph and \(M\) be a matching of \(G\).

	\noindent
	\textit{Sufficiency:}
	It will be proven by contradiction.
	Suppose \(M\) is a maximum matching and \(P\) is an \(M\)-augmenting path of \(G\).
	Note that, for \(i \in [l]\), one has: (1) \(e_i \in M\), if \(i\) is even; (2) \(e_i \not \in M\), if \(i\) is odd.

	Let \(M' \coloneqq M \symdiff E(P)\), \(M'\) is a matching since \(v_0, v_l\) are \(M\)-exposed and every vertex in \(\{v_1, \dots, v_{l-1}\}\) is covered by an edge in \(M \cap E(P)\).
	Hence, \(|M'| = |M|+1\), a contradiction. 

	\noindent
	\textit{Necessity:}
	Suppose \(G\) has no \(M\)-augmenting paths.
	Let \(M'\) be a maximum matching of \(G\) and \(G'\) be the graph induced by \(M \symdiff M'\).
	Note that \(G'\) has at least one component and every component of \(G'\) is either a path or a cycle.
	Let \(H\) be a component of \(G'\),
	\begin{enumerate}
		\item 
			if \(|V_H|\) is \textbf{even}, then 
				\(|M \cap V_H| = |M' \cap V_H|\);
		\item 
			if \(|V_H|\) is \textbf{odd}, then \(H\) must be a path. 
			Note that \(M'\) being maximum implies 
			\(|M \cap V_H| > |M' \cap V_H|\) 
			and, consequently,
			\(M' \cap V_H\) is a \(M\)-augmenting path of \(G\).
			Therefore, this case is impossible.
	\end{enumerate}
	Thus, \(|M| = |M'|\), i.e., \(M\) is a maximum matching.
\end{proof}

\begin{definition}[Vertex cover]
	\label{def:vertex_cover}
	For a graph \(G \coloneqq (V, E, \varphi)\), a subset \(K \subseteq V(G)\) is a \textbf{vertex cover} of \(G\) if every edge of \(E(G)\) has an end in \(K\). 
	A vertex cover is said to be \textbf{minimal} if one has \(|K| \leq |K'|\) for every vertex cover \(K'\) of \(G\).
	Denote \(\tau(G)\) as the size of a minimum vertex cover of \(G\).
\end{definition}

\begin{corollary}[Matching upperbound]
	\label{cor:match_upperbound}
	Let \(G\) be a graph, \(M\) be a matching of \(G\) and \(K\) be a vertex cover of \(G\). 
	Then, \(|M| \leq |K|\)..
\end{corollary}

\begin{proof}
	By definition, every edge of \(M\) has at least one end in \(K\) and no edge in \(M\) share an end.
	Hence, \(|K| \geq |M|\).
\end{proof}

\begin{theorem}[K\"onig's matching theorem]
	\label{thm:konig}
	Let \(G\) be a bipartite graph, then the maximum size of a matching of \(G\) is equal to the minimum size of a vertex cover of \(G\).
\end{theorem}

\begin{proof}
\end{proof}

Note that K\"onig's theorem \textbf{does not} hold for all graphs;
It suffices to consider a single odd cycle.

