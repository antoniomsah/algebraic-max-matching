\chapter{Blossom algorithm}

\enlargethispage{.5\baselineskip}

\section{Matchings}
\label{cap:grafos}

\begin{definition}[Graph]
	A \textbf{graph} \(G\) is a triple \((V, E, \varphi)\) such that
	\begin{enumerate}[label=(\roman*)]
		\item \(V\) is the \textbf{vertex set};
		\item \(E\) is the \textbf{edge set};
		\item \(\varphi: E \to V \times V\) is a relation between each edge and a pair of vertices, called the \textbf{incidence function} of \(G\).
	\end{enumerate}
\end{definition}

\begin{definition}[Walk]
	For a graph \(G \coloneqq (V, E, \varphi)\), a \textbf{walk} is a sequence
	\[
		\langle v_0, e_1, v_1, \dots, a_l, v_l \rangle \eqqcolon W
	\]
	such that
	\begin{enumerate}[label=(\roman*)]
		\item \(l \in \Naturals\) is the \textit{length} of \(W\);
		\item \(v_0, v_1, \dots, v_l \in V\);
		\item \(e_1, \dots, e_l \in E\).
	\end{enumerate}
	It is denoted that \(V(W) 
	\coloneqq \{v_0, \dots, v_l\}\) 
	and 
	\(E(W) \coloneqq \{e_1, \dots, e_l\}\). 
	It is said that \(W\) is walk from \(v_0\) to \(v_l\) (\((v_0, v_l)\)-walk).
	The walk \(W\) is a:
	\begin{itemize}
		\item 
			\textbf{path} if all its vertices are distinct;
		\item 
			\textbf{cycle} if \(v_0 = v_l\);
	\end{itemize}
\end{definition}

\begin{definition}[Matching]
	For a graph \(G \coloneqq (V, E, \varphi)\), a set \(M \subseteq E\) is a \textbf{matching} of \(G\) if and only if no two edges in \(M\) share an end.

	The matching \(M\) is:
	\begin{itemize}
		\item 
			\textbf{maximal} if there is no edge \(e \in E \setminus M\) such that \(M \cup \{e\}\) is a matching of \(G\);
		\item
			\textbf{maximum} if for every matching \(M'\) of \(G\) one has \(|M| \geq |M'|\).
	\end{itemize}
\end{definition}

Now, the maximum matching problem can be defined as follows:
\\
\begin{problem}{MaxMatching}
	Given a graph \(G \coloneqq (V, E, \varphi)\) find a maximum matching of \(G\).
\end{problem}

