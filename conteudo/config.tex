\usepackage{mdframed}

%%%%%%%%%%%%%% Definitions %%%%%%%%%%%%%%%%%%%%%%
\newcommand{\Naturals}{\mathbb{N}}
\newcommand{\Reals}{\mathbb{R}}

\newcommand{\SC}[1]{\textsc{#1}}

\newmdenv[linecolor=black,linewidth=1pt]{problembox}

\newenvironment{problem}[1]
{
    \begin{problembox}
    \begin{center}
        \Large\SC{#1}
    \end{center}
	\itshape
	\noindent
}
{
    \end{problembox}
}

%%%%%%%%%%%%% Theorems, lemmas, etc. %%%%%%%%%%%%

% Hints
% \newtheoremstyle{maybe} % nome do estilo
% {3pt} % espaço antes
% {3pt} % espaço depois
% {\itshape} % fonte do corpo
% {} % Indentação
% {\bfseries} % fonte do título
% {:} % pontuação após o título
% {.5em} % espaço após o título
% {\thmname{#1}\space(?)\thmnumber{ #2}\thmnote{ (#3)}} % formato do título
% % vazio significa {\thmname{#1}\thmnumber{ #2}\thmnote{ (#3)}}

%% Definition style
\newtheoremstyle{definition} 
{3pt} 
{3pt} 
{} 
{} 
{\bfseries} 
{:} 
{.5em} 
{\thesection.\thmnumber{#2}. \thmnote{#3}} 
\theoremstyle{definition}
\newtheorem{definition}{Definition}

\newtheoremstyle{corollary}
{3pt} 
{3pt} 
{\itshape} 
{} 
{\bfseries} 
{:} 
{.5em} 
{\thmname{#1} \thesection.\thmnumber{#2} (\thmnote{#3})} 
\theoremstyle{corollary}
\newtheorem{corollary}{Corollary}

\newtheoremstyle{theorem}
{3pt} 
{3pt} 
{\itshape} 
{} 
{\bfseries} 
{:} 
{.5em} 
{\thmname{#1} \thesection.\thmnumber{#2} (\thmnote{#3})} 
\theoremstyle{theorem}
\newtheorem{theorem}{Theorem}
