% Author: Nelson Lago
% This file is distributed under the MIT Licence

%%%%%%%%%%%%%%%%%%%%%%%%%%%%%%%%%%%%%%%%%%%%%%%%%%%%%%%%%%%%%%%%%%%%%%%%%%%%%%%%
%%%%%%%%%%%%%%%%%%%%%%%%%%%%%%%%% PREÂMBULO %%%%%%%%%%%%%%%%%%%%%%%%%%%%%%%%%%%%
%%%%%%%%%%%%%%%%%%%%%%%%%%%%%%%%%%%%%%%%%%%%%%%%%%%%%%%%%%%%%%%%%%%%%%%%%%%%%%%%

% A língua padrão é a última da lista
\documentclass[a1paper,brazilian,english]{article}

% Vários pacotes e opções de configuração genéricos
\usepackage{imegoodies}
\usepackage[poster,hidelinks]{imelooks}
\usepackage{mdframed}

\setlist{itemsep=0.5pt, parsep=0pt, topsep=0pt, partopsep=0pt}

%%%%%%%%%%%%%% Definitions %%%%%%%%%%%%%%%%%%%%%%
\newcommand{\Naturals}{\mathbb{N}}
\newcommand{\Integers}{\mathbb{Z}}
\newcommand{\Reals}{\mathbb{R}}
\newcommand*{\symdiff}{\mathbin{\Delta}}
\DeclareMathOperator{\Pf}{Pf}

\newcommand{\SC}[1]{\textsc{#1}}

\DeclareMathOperator{\rank}{rank}

\newmdenv[linecolor=black,linewidth=1pt]{problembox}

\newenvironment{problem}[1]
{
    \begin{problembox}
    \begin{center}
        \Large\SC{#1}
    \end{center}
	\itshape
	\noindent
}
{
    \end{problembox}
}

\newcommand\numberthis{\addtocounter{equation}{1}\tag{\theequation}}

%%%%%%%%%%%%% Theorems, lemmas, etc. %%%%%%%%%%%%

% Hints
% \newtheoremstyle{maybe} % nome do estilo
% {3pt} % espaço antes
% {3pt} % espaço depois
% {\itshape} % fonte do corpo
% {} % Indentação
% {\bfseries} % fonte do título
% {:} % pontuação após o título
% {.5em} % espaço após o título
% {\thmname{#1}\space(?)\thmnumber{ #2}\thmnote{ (#3)}} % formato do título
% % vazio significa {\thmname{#1}\thmnumber{ #2}\thmnote{ (#3)}}
\newtheorem{theorem}{Theorem}[section]
\newtheorem{theorem*}{Theorem}

% Lemma style
\newtheorem{lemma}[theorem]{Lemma}
\newtheorem{lemma*}{Lemma}

%% Definition style
\newtheorem{definition}[theorem]{Definition}
\newtheorem{definition*}{Definition}

% Corollary style
\newtheorem{corollary}[theorem]{Corollary}
\newtheorem{corollary*}{Corollary}

% Fact style
\newtheorem{fact}[theorem]{Fact}
\newtheorem{fact*}{Fact}

% Algorithm style
\newtheorem{algorithm}[theorem]{Algorithm}
\newtheorem{algorithm*}{Algorithm}

% Proof style
\makeatletter
\renewenvironment{proof}[1][\proofname]{\par
  \pushQED{\qed}%
  \normalfont \topsep0pt \partopsep0pt % Removes extra vertical space before proof
  \trivlist
  \item[\hskip\labelsep
        \itshape
    #1\@addpunct{.}]\ignorespaces
}{%
  \popQED\endtrivlist\@endpefalse
}
\makeatother

\BeforeBeginEnvironment{theorem}{\vspace{0.4\baselineskip}}
\BeforeBeginEnvironment{lemma}{\vspace{0.4\baselineskip}}
\BeforeBeginEnvironment{fact}{\vspace{0.4\baselineskip}}
\BeforeBeginEnvironment{proposition}{\vspace{0.4\baselineskip}}
\BeforeBeginEnvironment{corollary}{\vspace{0.4\baselineskip}}
\BeforeBeginEnvironment{definition}{\vspace{0.4\baselineskip}}
\BeforeBeginEnvironment{example}{\vspace{0.4\baselineskip}}
\BeforeBeginEnvironment{remark}{\vspace{0.4\baselineskip}}

%% LstListing C++ code

% \lstset{
%     language=cpp,
%     basicstyle=\ttfamily,
%     keywordstyle=\color{blue}\bfseries,
%     stringstyle=\color{red},
%     commentstyle=\color{green},
%     numbers=left,
%     numberstyle=\tiny,
%     stepnumber=1,
%     numbersep=5pt,
%     backgroundcolor=\color{white},
%     showspaces=false,
%     showstringspaces=false,
%     showtabs=false,
%     frame=single,
%     tabsize=2,
%     captionpos=b,
%     breaklines=true,
%     breakatwhitespace=false,
%     title=\lstname,
%     escapeinside={(*@}{@*)},
%     morekeywords={println, vector}
% }

% \tcbposterset{fontsize = 32pt} % default, mude se necessário

\graphicspath{{figuras/},{fig/},{conteudo/logos/},{img/},{images/},{imagens/},{content/images/}}

% Comandos rápidos para mudar de língua:
% \en -> muda para o inglês
% \br -> muda para o português
% \texten{blah} -> o texto "blah" é em inglês
% \textbr{blah} -> o texto "blah" é em português
\babeltags{br = brazilian, en = english}


%%%%%%%%%%%%%%%%%%%%%%%%%%%%%%%%%%%%%%%%%%%%%%%%%%%%%%%%%%%%%%%%%%%%%%%%%%%%%%%%
%%%%%%%%%%%%%%%%%%%%%%%%%%%%%%%%%% METADADOS %%%%%%%%%%%%%%%%%%%%%%%%%%%%%%%%%%%
%%%%%%%%%%%%%%%%%%%%%%%%%%%%%%%%%%%%%%%%%%%%%%%%%%%%%%%%%%%%%%%%%%%%%%%%%%%%%%%%

% O arquivo com os dados bibliográficos para biblatex; você pode usar
% este comando mais de uma vez para acrescentar múltiplos arquivos
\addbibresource{bibliografia.bib}

% Este comando permite acrescentar itens à lista de referências sem incluir
% uma referência de fato no texto (pode ser usado em qualquer lugar do texto)
%\nocite{bronevetsky02,schmidt03:MSc, FSF:GNU-GPL, CORBA:spec, MenaChalco08}
% Com este comando, todos os itens do arquivo .bib são incluídos na lista
% de referências
%\nocite{*}


%%%%%%%%%%%%%%%%%%%%%%%%%%%%%%%%%%%%%%%%%%%%%%%%%%%%%%%%%%%%%%%%%%%%%%%%%%%%%%%%
%%%%%%%%%%%%%%%%%%%%%%%%%%%%%%% INÍCIO DO POSTER %%%%%%%%%%%%%%%%%%%%%%%%%%%%%%%
%%%%%%%%%%%%%%%%%%%%%%%%%%%%%%%%%%%%%%%%%%%%%%%%%%%%%%%%%%%%%%%%%%%%%%%%%%%%%%%%


% Existem várias packages para criar pôsteres com LaTeX (a0poster, baposter,
% tikzposter, sciposter...). As mais comuns atualmente são beamerposter
% e tcolorbox (com sua biblioteca "poster"). Ambas funcionam muito bem;
% beamerposter é mais familiar (ela simplesmente utiliza beamer com alguns
% ajustes no tamanho das fontes e do papel), mas com tcolorbox o alinhamento
% vertical dos elementos é MUITO mais simples, e esta é a solução adotada
% aqui. Vale muito a pena ler a documentação com "texdoc tcolorbox" e
% "texdoc tcolorbox-tutorial-poster".

% Um pôster com tcolorbox é composto por blocos (posterboxes) coloridos
% de tamanho variável; cada bloco pode conter textos ou imagens e um
% título opcional. O pôster utiliza uma grade de dimensões definidas em
% \begin{tcposter} com "rows=" e "columns=" para fazer o alinhamento:
% para cada posterbox, podemos dizer "row=X, column=Y" para definir sua
% posição. Além disso, podemos dizer "span=A, rowspan=B" para fixar
% seu tamanho. Sem "span" e "rowspan", uma posterbox tem pelo menos o
% tamanho de uma célula da grade, mas se seu tamanho natural for maior
% ela extrapola esse tamanho. "span" e "rowspan" podem ser números
% não-inteiros (como 0.8 ou 1.4).
%
% "\begin{posterbox}" recebe um conjunto de parâmetros opcional e um
% conjunto de parâmetros obrigatório:
%
% "\begin{posterbox}[opcional]{obrigatório}".
%
% O conjunto de parâmetros opcional é onde inserimos os parâmetros comuns
% de tcolorbox, como "adjusted title", "coltext", "titlerule" etc.; o
% conjunto de parâmetros obrigatório é usado para determinar as dimensões
% e a posição da posterbox, ou seja, as opções "name", "column", "below",
% "span" etc.
%
% ALINHAMENTO HORIZONTAL
%
% É possível definir um poster com 2 colunas e fazer algo como
%
% \posterbox{column=1, span=1.3}{blah}
% \posterbox{column*=2, span=0.7}{blah}
%
% A segunda posterbox será alinhada à direita ("column*="), então as
% duas serão colocadas lado-a-lado sem sobreposições.
%
% Na prática, no entanto, é mais fácil fazer como no exemplo abaixo:
% definimos que o poster tem 12 colunas, o que nos permite dividir
% sua largura em 2, 3, 4 ou 6 colunas iguais ou diferentes (como
% 1/2 + 1/2, 2/3 + 1/3, 1/4 + 1/4 + 1/2, 1/4 + 1/6 + 1/4 + 1/3 etc).
%
% ALINHAMENTO VERTICAL
%
% Embora seja possível alinhar as posterboxes em função da grade na
% vertical, uma outra possibilidade é utilizar "above", "below" e
% "between", como no exemplo abaixo: basta associar um nome "blah" a
% uma determinada posterbox e, em outra, dizer "below=blah". Lembre-se
% que a posterbox de nome "blah" deve ser definida *antes* que outra
% possa fazer referência a ela. Também é possível fazer "below=top",
% "above=bottom" etc. A opção "equal height group" também é muito útil.
% Nada impede que você use estratégias de alinhamento diferentes para
% cada posterbox.

% Este modelo define a opção "smallmargins", que diminui a distância
% entre o conteúdo de uma posterbox e suas bordas. Use com parcimônia!

\begin{document}

% Em um poster não há \maketitle

\begin{tcbposter}[
  poster = {
    %showframe, % muito útil durante a preparação do poster
    rows = 6,
    columns = 12,
    colspacing = 1.2cm,
    rowspacing = .8cm,
  },
]

\posterbox[titlebox]{name=titlebox, below=top, column=1, span=12}{
  An Algebraic Algorithm for Maximum Matchings
}

\posterbox{name=subtitlebox, below=titlebox, column=1, span=12}{
  Author: Antonio Marcos Shiro Arnauts Hachisuca \\
  Supervisor: Marcel K. de Carli Silva
}

%%%%%%%% Duas colunas %%%%%%%%

% Como temos 2 caixas à esquerda e uma caixa à direita, não podemos
% simplesmente usar "equal height group" aqui, então definimos
% manualmente a altura das caixas de maneira que as duas colunas
% tenham o mesmo tamanho.

%%% Esquerda
\posterbox[adjusted title = Graph, smallmargins]
          {name=graph, below=subtitlebox,
           column=1, span=6, rowspan=1.3}{
  A \textbf{graph} \(G\) is a pair \((V,E)\) such that
    \begin{enumerate}
      \item \(V\) is a finite set, whose elements are called \textbf{vertices};
      \item \(E\) is a set of unordered pairs of vertices, whose elements are called \textbf{edges}.
    \end{enumerate}
  The image below illustrates a graph, where the circles represent the vertices, and the lines represent the edges.
  \begin{center}
    \begin{tikzpicture}[scale=3, every node/.style={circle, draw, inner sep=2.5pt}]

    % Outer cycle vertices
    \node (A) at (90:1) {};
    \node (B) at (162:1) {};
    \node (C) at (234:1) {};
    \node (D) at (306:1) {};
    \node (E) at (18:1) {};
    
    % Inner star vertices
    \node (F) at (90:0.5) {};
    \node (G) at (162:0.5) {};
    \node (H) at (234:0.5) {};
    \node (I) at (306:0.5) {};
    \node (J) at (18:0.5) {};
    
    % Edges
    % Outer cycle
    \draw (A) -- (B) -- (C) -- (D) -- (E) -- (A);
    
    % Inner star
    \draw (F) -- (G) -- (H) -- (I) -- (J) -- (F);
    
    % Connecting edges
    \draw (A) -- (F);
    \draw (B) -- (G);
    \draw (C) -- (H);
    \draw (D) -- (I);
    \draw (E) -- (J);
    
    \end{tikzpicture}
  \end{center}
}
\posterbox[adjusted title = Matching, smallmargins]
          {name=matching, below=graph,
           column=1, span=6, rowspan=2.15}{

    
  A \textbf{matching} \(M\) of a graph \(G\) is a subset of edges from \(G\) such that no two edges share an end.
  There are a few classifications for matchings: A matching is \textbf{maximum} if there is no other matching whose size is greater;
  A matching is \textbf{perfect} if every vertex touches an edge from the matching.
  \\~\\
  The red edges in the graphs below represent two different matchings.
  \begin{center}
    \begin{tikzpicture}[scale=3, every node/.style={circle, draw, inner sep=2.5pt}]

    % Outer cycle vertices
    \begin{scope}
    \node (A) at (90:1) {};
    \node (B) at (162:1) {};
    \node (C) at (234:1) {};
    \node (D) at (306:1) {};
    \node (E) at (18:1) {};

    \draw (A) -- (B);
    \draw[red, ultra thick] (B) -- (C);
    \draw (C) -- (D);
    \draw[red, ultra thick] (D) -- (E);
    \draw (E) -- (A);
    \end{scope}
    
    \begin{scope}[xshift=3cm]
    % Outer cycle vertices
    \node (A) at (90:1) {};
    \node (B) at (162:1) {};
    \node (C) at (234:1) {};
    \node (D) at (306:1) {};
    \node (E) at (18:1) {};
    
    % Inner star vertices
    \node (F) at (90:0.5) {};
    \node (G) at (162:0.5) {};
    \node (H) at (234:0.5) {};
    \node (I) at (306:0.5) {};
    \node (J) at (18:0.5) {};
    
    % Edges
    % Outer cycle
    \draw (A) -- (B);
    \draw[red, ultra thick] (B) -- (C);
    \draw (C) -- (D);
    \draw[red, ultra thick] (D) -- (E);
    \draw (E) -- (A);
    
    % Inner star
    \draw (F) -- (G);
    \draw[red, ultra thick] (G) -- (H);
    \draw (H) -- (I);
    \draw[red, ultra thick] (I) -- (J);
    \draw (J) -- (F);
    
    % Connecting edges
    \draw[red, ultra thick] (A) -- (F);
    \draw (B) -- (G);
    \draw (C) -- (H);
    \draw (D) -- (I);
    \draw (E) -- (J);
    \end{scope}
    
    \end{tikzpicture}
  \end{center}
  Note that the matching on the left is a \textbf{maximum} but not \textbf{perfect} matching, whereas the matching on the right is a \textbf{perfect} matching.
}

\posterbox[adjusted title = Tutte Matrix, smallmargins]{
  name=TutteMatrix, below=matching, column=1, span=6, rowspan=1.65
}{
The \textbf{Tutte Matrix} of a graph is a matrix where every edge is assigned an indeterminate.
For example, 
\begin{center}
  \begin{tikzpicture}[scale=2, every node/.style={inner sep=3pt}]

    % Define vertices
    \node[draw, circle] (A) at (0, 0) {};  % Bottom-left
    \node[draw, circle] (B) at (2, 0) {};  % Bottom-right
    \node[draw, circle] (C) at (2, 2) {};  % Top-right
    \node[draw, circle] (D) at (0, 2) {};  % Top-left
    
    % Draw edges
    \draw (A) -- (B);
    \draw (B) -- (C);
    \draw (C) -- (D);
    \draw (D) -- (A);
    
    % Add an arrow pointing to the adjacency matrix
    \draw[->, thick] (2.25, 1) -- (3, 1);
    
    % Draw the adjacency matrix
    \node[anchor=west] at (3.25, 1) {$
    \begin{bmatrix}
    0 & a & 0 & b \\
    -a & 0 & c & 0 \\
    0 & -c & 0 & d \\
    -b & 0 & -d & 0
    \end{bmatrix}
    $};
  \end{tikzpicture}
\end{center}
\begin{fact*}
  Let \(G\) be a graph, \(G\) has a perfect matching if and only if the Tutte matrix of \(G\) is non-singular.
\end{fact*}
Unfortunately, it isnt viable to represent a Tutte matrix with indeterminates.
Because, the number of monomials is proportial to the number of perfect matchings of \(G\), and there can be \textbf{exponentially} many of those.
}

\posterbox{
  name=TutteMatrix, below=subtitlebox, column=7, span=6, rowspan=1.65
}{
  Fortunately, \citep{Lovasz:Random} presented a probabilistic solution:
  If we replace the non-zero entries of (T) with random values from a sufficiently large field, the matrix's rank is preserved with \textbf{high} probability. 
}


% %%%%%%% Colunas assimétricas (2/3 + 1/3) %%%%%%%%
% 
% %%% Esquerda. Como usamos "between", a altura desta caixa
% %             é definida pela posição das outras duas.
% \posterbox[adjusted title = Bibliography (assimetrical
%            columns -- two thirds width), smallmargins]
%           {name=bibbox, between=halflogo and footerbox, column=1, span=8}{
% 
%     \nocite{FSF:GNU-GPL, MenaChalco08, biblatex,
%             waz:09, alves03:simi, carlis:09}
%     \printbibliography
% }
% 
% %%% Direita, acima; aqui, ajustamos as margens manualmente
% \posterbox[adjusted title = A Table (one third width),
%           top=.5\baselineskip, bottom=.5\baselineskip]
%           {name=tablebox, below=, column=9, span=4}{
% 
%     \centering
%     \singlespacing\vspace{-\baselineskip} % \singlespacing adiciona uma linha
%     \begin{tabular}{ccl}
%       \toprule
%       Code        & Abbreviation & Name       \\
%       \midrule
%       \texttt{A}  & Ala          & Alanine    \\
%       \texttt{C}  & Cys          & Cysteine   \\
%       \texttt{W}  & Trp          & Tryptophan \\
%       \texttt{Y}  & Tyr          & Tyrosine   \\
%       \bottomrule
%     \end{tabular}
% }

\end{tcbposter}

\end{document}
