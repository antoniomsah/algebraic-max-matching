% Author: Nelson Lago
% This file is distributed under the MIT Licence

\documentclass[a1paper,brazilian,english]{article}

\usepackage{imegoodies}
\usepackage[poster,hidelinks,bluey]{imelooks}
\usepackage{mdframed}

\setlist{itemsep=0.5pt, parsep=0pt, topsep=0pt, partopsep=0pt}

%%%%%%%%%%%%%% Definitions %%%%%%%%%%%%%%%%%%%%%%
\newcommand{\Naturals}{\mathbb{N}}
\newcommand{\Integers}{\mathbb{Z}}
\newcommand{\Reals}{\mathbb{R}}
\newcommand*{\symdiff}{\mathbin{\Delta}}
\DeclareMathOperator{\Pf}{Pf}

\newcommand{\SC}[1]{\textsc{#1}}

\DeclareMathOperator{\rank}{rank}

\newmdenv[linecolor=black,linewidth=1pt]{problembox}

\newenvironment{problem}[1]
{
    \begin{problembox}
    \begin{center}
        \Large\SC{#1}
    \end{center}
	\itshape
	\noindent
}
{
    \end{problembox}
}

\newcommand\numberthis{\addtocounter{equation}{1}\tag{\theequation}}

%%%%%%%%%%%%% Theorems, lemmas, etc. %%%%%%%%%%%%

% Hints
% \newtheoremstyle{maybe} % nome do estilo
% {3pt} % espaço antes
% {3pt} % espaço depois
% {\itshape} % fonte do corpo
% {} % Indentação
% {\bfseries} % fonte do título
% {:} % pontuação após o título
% {.5em} % espaço após o título
% {\thmname{#1}\space(?)\thmnumber{ #2}\thmnote{ (#3)}} % formato do título
% % vazio significa {\thmname{#1}\thmnumber{ #2}\thmnote{ (#3)}}
\newtheorem{theorem}{Theorem}[section]
\newtheorem{theorem*}{Theorem}

% Lemma style
\newtheorem{lemma}[theorem]{Lemma}
\newtheorem{lemma*}{Lemma}

%% Definition style
\newtheorem{definition}[theorem]{Definition}
\newtheorem{definition*}{Definition}

% Corollary style
\newtheorem{corollary}[theorem]{Corollary}
\newtheorem{corollary*}{Corollary}

% Fact style
\newtheorem{fact}[theorem]{Fact}
\newtheorem{fact*}{Fact}

% Algorithm style
\newtheorem{algorithm}[theorem]{Algorithm}
\newtheorem{algorithm*}{Algorithm}

% Proof style
\makeatletter
\renewenvironment{proof}[1][\proofname]{\par
  \pushQED{\qed}%
  \normalfont \topsep0pt \partopsep0pt % Removes extra vertical space before proof
  \trivlist
  \item[\hskip\labelsep
        \itshape
    #1\@addpunct{.}]\ignorespaces
}{%
  \popQED\endtrivlist\@endpefalse
}
\makeatother

\BeforeBeginEnvironment{theorem}{\vspace{0.4\baselineskip}}
\BeforeBeginEnvironment{lemma}{\vspace{0.4\baselineskip}}
\BeforeBeginEnvironment{fact}{\vspace{0.4\baselineskip}}
\BeforeBeginEnvironment{proposition}{\vspace{0.4\baselineskip}}
\BeforeBeginEnvironment{corollary}{\vspace{0.4\baselineskip}}
\BeforeBeginEnvironment{definition}{\vspace{0.4\baselineskip}}
\BeforeBeginEnvironment{example}{\vspace{0.4\baselineskip}}
\BeforeBeginEnvironment{remark}{\vspace{0.4\baselineskip}}

%% LstListing C++ code

% \lstset{
%     language=cpp,
%     basicstyle=\ttfamily,
%     keywordstyle=\color{blue}\bfseries,
%     stringstyle=\color{red},
%     commentstyle=\color{green},
%     numbers=left,
%     numberstyle=\tiny,
%     stepnumber=1,
%     numbersep=5pt,
%     backgroundcolor=\color{white},
%     showspaces=false,
%     showstringspaces=false,
%     showtabs=false,
%     frame=single,
%     tabsize=2,
%     captionpos=b,
%     breaklines=true,
%     breakatwhitespace=false,
%     title=\lstname,
%     escapeinside={(*@}{@*)},
%     morekeywords={println, vector}
% }


\graphicspath{{figuras/},{fig/},{conteudo/logos/},{img/},{images/},{imagens/},{content/images/}}

\babeltags{br = brazilian, en = english}

\addbibresource{poster.bib}

\begin{document}

% Em um poster não há \maketitle

\begin{tcbposter}[
  poster = {
    %showframe, % muito útil durante a preparação do poster
    rows = 6,
    columns = 12,
    colspacing = 1.2cm,
    rowspacing = .8cm,
  },
]

\posterbox[titlebox]{name=titlebox, below=top, column=1, span=12}{
  \begin{center}
    \huge{\textbf{An Algebraic Algorithm for Maximum Matchings}}
  \end{center}
  \begin{multicols}{2}
    \begin{center}
      Author: Antonio Marcos Shiro Arnauts Hachisuca
    \end{center}

    \begin{center}
      Supervisor: Marcel K. de Carli Silva
    \end{center}
  \end{multicols}
}

%%% Esquerda
\posterbox[adjusted title = Matching, smallmargins]
          {name=matching, below=titlebox,
           column=1, span=6, rowspan=1.45}{

    
  A \textbf{matching} \(M\) of a graph \(G\) is a subset of edges from \(G\) such that no two edges share an end.
  A matching is \textbf{maximum} if there is no other matching whose size is greater;
  A matching is \textbf{perfect} if every vertex touches an edge from the matching.
  \\~\\
  The red edges in the graphs below represent two different matchings.
  \begin{center}
    \begin{tikzpicture}[scale=3, every node/.style={circle, draw, inner sep=2.5pt}]

    % Outer cycle vertices
    \begin{scope}
    \node (A) at (90:1) {};
    \node (B) at (162:1) {};
    \node (C) at (234:1) {};
    \node (D) at (306:1) {};
    \node (E) at (18:1) {};

    \draw (A) -- (B);
    \draw[red, ultra thick] (B) -- (C);
    \draw (C) -- (D);
    \draw[red, ultra thick] (D) -- (E);
    \draw (E) -- (A);
    \end{scope}
    
    \begin{scope}[xshift=3cm]
    % Outer cycle vertices
    \node (A) at (90:1) {};
    \node (B) at (162:1) {};
    \node (C) at (234:1) {};
    \node (D) at (306:1) {};
    \node (E) at (18:1) {};
    
    % Inner star vertices
    \node (F) at (90:0.5) {};
    \node (G) at (162:0.5) {};
    \node (H) at (234:0.5) {};
    \node (I) at (306:0.5) {};
    \node (J) at (18:0.5) {};
    
    % Edges
    % Outer cycle
    \draw (A) -- (B);
    \draw[red, ultra thick] (B) -- (C);
    \draw (C) -- (D);
    \draw[red, ultra thick] (D) -- (E);
    \draw (E) -- (A);
    
    % Inner star
    \draw (F) -- (G);
    \draw[red, ultra thick] (G) -- (H);
    \draw (H) -- (I);
    \draw[red, ultra thick] (I) -- (J);
    \draw (J) -- (F);
    
    % Connecting edges
    \draw[red, ultra thick] (A) -- (F);
    \draw (B) -- (G);
    \draw (C) -- (H);
    \draw (D) -- (I);
    \draw (E) -- (J);
    \end{scope}
    
    \end{tikzpicture}
  \end{center}
  Note that the matching on the left is a \textbf{maximum} but not perfect matching, whereas the matching on the right is a \textbf{perfect} matching.

  Now, there is the following problem:
  \begin{problem}{Maximum Matching}
    Given a graph \(G\), find a maximum matching of \(G\).
  \end{problem}
}

\posterbox[adjusted title = Linear Algebra, smallmargins]{
  name=LinearAlgebra, below=matching, column=1, span=6, rowspan=0.98
}{
  \textbf{Submatrix notation used: } For a matrix \(M\), \(M_{R, S}\) is the submatrix of \(M\) with 

  rows \(R\) and columns \(S\). \(M_{*, S}\) denotes the submatrix of \(M\) with all rows and 
  
  columns \(S\) (resp. \(M_{R, *}\)).
  \\~\\
  The following corollary is used to update a matrix inverse.
  \begin{corollary*}[Simplified version of corollary 2.1 from Harvey's paper]
    \label{cor:2.1}
    Let \(M\) be a non-singular matrix and let \(N\) be its inverse.
    Let \(\tilde{M}\) be a matrix that is identical to \(M\) except \(M_{S, S} \neq \tilde{M}_{S, S}\) for a set \(S\).
    Let \(\Delta \coloneqq \tilde{M}_{S, S} - M_{S, S}\).
    Then, if \(\tilde{M}\) is non-singular, one has
    \[
      \tilde{M}^{-1} = N - N_{*, S}(I + \Delta N_{S, S})^{-1}\Delta N_{S, *}.
    \]
  \end{corollary*}
}

\posterbox[adjusted title = Tutte Matrix, smallmargins]{
  name=TutteMatrix, below=LinearAlgebra, column=1, span=6, rowspan=2.77
}{
An \textbf{indeterminate} is a symbol or expression denoting an unspecified value.
Unlike a variable, it maintains a fixed, inherently unique identity (meaning two indeterminates, \(a\) and \(b\), can never be equal).
\\~\\
The \textbf{Tutte Matrix} of a graph \(G\), denoted as \(T_G\), is a skew-symmetric matrix (i.e. \(T_G = -T_G^{\mathsf{T}}\)) where every edge is assigned an indeterminate.
For example, 
\begin{center}
  \begin{tikzpicture}[scale=2, every node/.style={inner sep=3pt}]

    % Define vertices
    \node[draw, circle] (A) at (0, 0) {};  % Bottom-left
    \node[draw, circle] (B) at (2, 0) {};  % Bottom-right
    \node[draw, circle] (C) at (2, 2) {};  % Top-right
    \node[draw, circle] (D) at (0, 2) {};  % Top-left
    
    % Draw edges
    \draw (A) -- (B);
    \draw (B) -- (C);
    \draw (C) -- (D);
    \draw (D) -- (A);
    
    % Add an arrow pointing to the adjacency matrix
    \draw[->, thick] (2.25, 1) -- (3, 1);
    
    % Draw the adjacency matrix
    \node[anchor=west] at (3.25, 1) {$
    \begin{bmatrix}
    0 & a & 0 & b \\
    -a & 0 & c & 0 \\
    0 & -c & 0 & d \\
    -b & 0 & -d & 0
    \end{bmatrix}
    $};
  \end{tikzpicture}
  \\
  \textit{\scriptsize Figure 1: Example of a graph and its corresponding Tutte Matrix.}
\end{center}
\begin{fact*}[\citep{Tutte}]
  Let \(G\) be a graph, \(G\) has a perfect matching iff \(T_G\) is non-singular.
\end{fact*}

\begin{center}
\begin{tcolorbox}[width=0.95\textwidth, top=-5mm, bottom=-5mm]
  \textbf{Reminder: } A matrix is singular if and only if its determinant is zero.
\end{tcolorbox}
\end{center}

\textbf{Computational Challenge:} While this fact provides a necessary and sufficient condition for the existence of a perfect matching, it does not immediately yield a polynomial-time algorithm to find one. 
The determinant of a Tutte Matrix has a monomial for each perfect matching of \(G\),
and the number of perfect matchings can grow \emph{exponentially} with the graph's size.

\textbf{Probabilistic solution:} \citet{Lovasz:Random} proposed a probabilistic solution:
\begin{itemize}
  \item Replace the indeterminates of \(T_G\) with random values from a large field.
\end{itemize}

\begin{lemma*}[Schwartz-Zippel]
Let \(\mathbb{F}\) be a field, and let \(p(x_1, x_2, \dots, x_n)\) be a non-zero polynomial in \(\mathbb{F}[x_1, x_2, \dots, x_n]\) of total degree \(d\). 
Suppose \(S \subseteq \mathbb{F}\) is a finite subset of the field \( \mathbb{F} \). 
If the variables \(x_1, x_2, \dots, x_n\) are chosen independently and uniformly at random from \( S \), then

\[
\Pr\big[p(x_1, x_2, \dots, x_n) = 0\big] \leq \frac{d}{|S|}.
\]
\end{lemma*}

This approach together with Schwartz-Zippel lemma is a solution that fails 

with \textbf{low} probability.

\textbf{How to select the field:} For a prime \(p\), \(\Integers_{p}\) is a field of size \(|p|\).
Thus, choosing a sufficiently large field is reduced to choosing a sufficiently large prime.
}

\posterbox[smallmargins]{
  name=TutteMatrix2, below=titlebox, column=7, span=6, rowspan=0.71
}{
Nonetheless, it is \textbf{probabilistic}. 
For example, the Tutte matrix from Figure 1 can be represented with the following values from \(\Integers_{53}\):
\[
  \begin{bmatrix}
  0 & a & 0 & b \\
  -a & 0 & c & 0 \\
  0 & -c & 0 & d \\
  -b & 0 & -d & 0
  \end{bmatrix}
  \longrightarrow
  \begin{bmatrix}
  0 & 7 & 0 & 28 \\
  46 & 0 & 14 & 0 \\
  0 & 39 & 0 & 50 \\
  25 & 0 & 3 & 0
  \end{bmatrix}
\]
Note that this matrix is singular! \textbf{But}, the graph from Figure 1 \textbf{has} 

a perfect matching.
}

\posterbox[adjusted title = Matrix Algorithms, smallmargins]{
  name=Matrix, below=TutteMatrix2, column=7, span=6, rowspan=0.66
}{
  Let \(\omega\) denote the \textbf{matrix multiplication exponent}, defined as the \textbf{smallest} exponent such that an \(O(n^\omega)\) matrix multiplication algorithm exists.
  Then,
  \begin{itemize}
    \item There is a \textbf{matrix inversion algorithm} with \(O(n^\omega)\) time complexity;
    \item There is a \textbf{matrix rank algorithm} with \(O(n^\omega)\) time complexity.
  \end{itemize}

  \begin{center}
  \begin{tcolorbox}[width=0.95\textwidth, top=-5mm, bottom=-5mm]
    \textbf{Important: } Currently, it is known that \(\omega < 2.38\).
  \end{tcolorbox}
  \end{center}
}

\posterbox[adjusted title = Perfect Matching Algorithms, smallmargins]{
  name=PerfectMatching, below=Matrix, column=7, span=6, rowspan=2
}{
  Now, algorithms can be introduced. 
  The more direct one is the following:
  \begin{algorithm*}[Naive Algorithm]
    Given a graph \(G \coloneqq (V, E)\). 
    Set \(M \coloneqq \emptyset\).
    For each edge \(uv \in E\), check if a Tutte matrix of graph \(G' \coloneqq (V, E \setminus \{uv\})\) is singular.
    If \(T_{G'}\) is singular, then \(M \leftarrow M \cup \{uv\}\) and \(G \leftarrow G - \{u, v\}\); Else, \(G \leftarrow G'\).
    After all iterations, return \(M\).
  \end{algorithm*}
  The algorithm above yields a time complexity of \(O(n^{\omega + 2})\), 
  because a graph has \(O(n^2)\) edges and it checks if a matrix is singular in \(O(n^\omega)\) for each edge.

  The main bottleneck of the algorithm above is the necessity to recalculate the inverse for each edge check.

  \begin{corollary*}[Edge removal condition]
  \label{cor:edge_rem}
    Let \(G \coloneqq (V, E)\) be a graph such that \(G\) has a perfect matching.
    Let \(T\) be the Tutte Matrix of \(G\) and let \(N \coloneqq T^{-1}\).
    Then, an edge \(uv \in E\) can be removed from \(G\) (i.e. graph \((V, E \setminus \{uv\})\) has a perfect matching) iff 
    \[
      N_{uv} \neq -1/T_{uv}.
    \]
  \end{corollary*}

  By using corollaries \ref{cor:2.1} and \ref{cor:edge_rem}, \citet{Harvey:Paper} developed an algorithm that computes the perfect matching in \(O(n^\omega)\).
  The algorithm uses a \textbf{divide and conquer approach} together with \textbf{lazy updates} to find a perfect matching.
  \\~\\
  It maintains two matrices, \(T\) and \(N\), such that they are initially set as
  \begin{itemize}
    \item \(T \coloneqq \) a Tutte matrix of graph \(G\) where the entries are randomly chosen;
    \item \(N \coloneqq T^{-1}\).
  \end{itemize}
  By using \textbf{lazy updates}, for an edge \(uv \in E\), one has \(T_{uv}^{-1} = N_{uv}\) when the value is \textbf{needed} (this value \textbf{can be incorrect} while not required).
  This approach ensures that \cref{cor:edge_rem} holds when checking an edge \(uv\).
  Then, when an edge is removed it uses divide and conquer and \cref{cor:2.1} to update \textbf{only} the required submatrix.
}

\posterbox[adjusted title = Extension to Maximum Matchings, smallmargins]{
  name=MaxMatching, below=PerfectMatching, column=7, span=6, rowspan=1.05
}{
  The following theorem is the basis for the extension.
  \begin{theorem*}[\citet{plummer1986matching}]
  \label{thm:rank_matching}
    Let \(G\) be a graph and let \(\nu(G)\) be the size of a maximum matching of \(G\).
    Then, \(\rank(T_G) = 2\nu(G)\).
    \end{theorem*}
    The extension to maximum matching proceeds by first determining the rank of the Tutte matrix \(T_G\). 
    The number of unmatched vertices is calculated as \(|V| - \rank(T_G)\).
    To obtain a graph with a perfect matching, we add \(|V| - \rank(T_G)\) new vertices each of which 
    is connected to \textbf{all} original vertices of \(G\).
    \\~\\
    This new graph construction guarantees the existence of a perfect matching, because of \cref{thm:rank_matching}.
    Computing the matrix rank can be done in \(O(n^\omega)\) and building the auxiliary graph can be done in \(O(n^2)\) (\(O(n)\) for each of the \(n\) vertices).
}

\posterbox[adjusted title = References, smallmargins]{
  name=References, below=MaxMatching, column=7, span=6, rowspan=0.78
}{
  \printbibliography
}

\posterbox[footerbox]{name=footerbox, above=bottom, column=1, span=12}{
    linux.ime.usp.br/\textasciitilde amsah/mac0499\par
    \small\ttfamily
    amsah@usp.br\par
    \vspace{4pt}
    \footnotesize\rmfamily
    \textcolor{imesoftblue!30!white}
      {Department of Computer Science --- Institute of Mathematics and Statistics, University of São Paulo}
}


\end{tcbposter}

\end{document}
