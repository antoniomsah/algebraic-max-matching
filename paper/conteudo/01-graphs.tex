\renewcommand*{\proofname}{Proof}

\chapter{Blossom algorithm}

The main algorithm\footnote{If the graph is \textbf{guaranteed} to be bipartite, one may search \textit{Kuhn's algorithm} and \textit{Flow Networks}.} to solve the \textbf{maximum matching} problem is the algorithm \textbf{Blossom} developed by Jack Edmonds in 1965.
Hence, this chapter intends to: 
(1) introduce graph concepts relevant to solving the maximum matching problem;
(2) Prove the Blossom algorithm; 
(3) Implement the Blossom algorithm and 
(4) present applications of the Blossom algorithm.

\enlargethispage{.5\baselineskip}

\section{Matchings}

\begin{definition}[Graph]
	\label{def:graph}
	A \textbf{graph} \(G\) is a triple \((V, E, \varphi)\) such that
	\begin{enumerate}[label=(\roman*)]
		\item \(V\) is the \textbf{vertex set};
		\item \(E\) is the \textbf{edge set};
		\item \(\varphi: E \to V \times V\) is a relation between each edge and a pair of vertices, called the \textbf{incidence function} of \(G\).
	\end{enumerate}
	Usually, it is used 
	\(V(G)\) or \(V_G\) to denote \(V\) and 
	\(E(G)\) or \(E_G\) to denote \(E\).
	Also, if \(e \in E(G)\) and \(\varphi(e) = (u, v)\), then \(u\) and \(v\) are the \textbf{ends} of \(e\);
	When the context is clear, \((u, v)\) may be abbreviated to \(uv\).
\end{definition}

\begin{definition}[Subgraph]
	\label{def:subgraph}
	A graph \(H\) is a \textbf{subgraph} of a graph \(G\) if 
	\(V_H \subseteq V_G, E_H \subseteq E_G\) and every edge in \(E_H\) has the same ends in \(H\) and \(G\).
\end{definition}

\begin{definition}[Walk]
	\label{def:walk}
	For a graph \(G \coloneqq (V, E, \varphi)\), a \textbf{walk} is a sequence
	\[
		\langle v_0, e_1, v_1, \dots, a_l, v_l \rangle \eqqcolon W
	\]
	such that
	\begin{enumerate}[label=(\roman*)]
		\item \(l \in \Naturals\) is the \textit{length} of \(W\);
		\item \(v_0, v_1, \dots, v_l \in V\);
		\item \(e_1, \dots, e_l \in E\).
	\end{enumerate}
	It is denoted that \(V(W) 
	\coloneqq \{v_0, \dots, v_l\}\) 
	and 
	\(E(W) \coloneqq \{e_1, \dots, e_l\}\). 
	It is said that \(W\) is walk from \(v_0\) to \(v_l\) or a \((v_0, v_l)\)-walk.
	The walk \(W\) is a:
	\begin{itemize}
		\item 
			\textbf{path}, if all its vertices are distinct;
		\item 
			\textbf{cycle}, if \(v_0 = v_l\) 
			and it is an \textbf{odd cycle} if its length is odd, else it is an \textbf{even cycle}.
	\end{itemize}
	A vertex \(u \in V\) \textbf{reaches} \(v \in V\) if there is a \((u,v)\)-walk in \(G\). 
\end{definition}

\begin{definition}[Components]
	\label{def:components}
	A \textbf{(connected) component} of \(G\) is a subgraph \(H\) such that every vertex of \(V_H\) reaches every vertex of \(V_H\), but does not reach any vertex in \(V_G \setminus V_H\).
	If \(G\) has \textbf{exactly} one component, then \(G\) is \textbf{connected}; Else, \(G\) is \textbf{disconnected}.
\end{definition}


\begin{definition}[Bipartite graphs]
	A graph \(G\) is \textbf{bipartite} if there are two sets \(U, W \in V(G)\) such that 
	\begin{enumerate}[label=(\roman*)]
		\item \(U \cap W = \emptyset\);
		\item \(U \cup W = V(G)\);
		\item every edge of \(G\) has one end at \(U\) and the other end at \(W\).
	\end{enumerate}
	In this case, it is said that \(G\) is \((U, W)\)-bipartite.
\end{definition}

\begin{theorem}[Characterization of bipartite graphs]
	A graph is bipartite if and only if it has no odd cycles.
\end{theorem}

\begin{proof}

\end{proof}

\begin{definition}[Matching]
	\label{def:matching}
	For a graph \(G \coloneqq (V, E, \varphi)\), a set \(M \subseteq E\) is a \textbf{matching} of \(G\) if and only if no two edges in \(M\) share an end.
	A vertex \(v \in V\) is \(M\)-covered if some edge of \(M\) incides in \(v\), 
	and it is said that \(M\) covers \(v\);
	Otherwise, \(v\) is \(M\)-exposed.
	The matching \(M\) is:
	\begin{itemize}
		\item 
			\textbf{maximal}, if there is no edge \(e \in E \setminus M\) such that \(M \cup \{e\}\) is a matching of \(G\);

		\item
			\textbf{maximum}, if for every matching \(M'\) of \(G\) one has \(|M| \geq |M'|\);
	
		\item
			\textbf{perfect}, if \(2|V_G| = |M|\), i.e., every vertex of \(G\) is covered.
	\end{itemize}
	Denote \(\nu(G)\) as the size of a maximum matching in \(G\).
\end{definition}

