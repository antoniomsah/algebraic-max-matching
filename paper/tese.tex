% A opção twoside (frente-e-verso) significa que a aparência das páginas pares
% e ímpares pode ser diferente. Por exemplo, as margens podem ser diferentes ou
% os números de página podem aparecer à direita ou à esquerda alternadamente.
% Mas nada impede que você crie um documento "só frente" e, ao imprimir, faça
% a impressão frente-e-verso.
%
% Aqui também definimos a língua padrão do documento (a última da lista) e
% línguas adicionais. Para teses do IME, no mínimo português e inglês são
% obrigatórios, porque independentemente da língua principal do texto é
% preciso fornecer o resumo nessas duas línguas. LaTeX aceita alguns nomes
% diferentes para a língua portuguesa; dentre as opções, prefira sempre
% "brazilian" para português brasileiro e "portuguese" para português europeu.
\documentclass[a4paper,12pt,twoside,brazilian,english]{book}
%\documentclass[a4paper,12pt,twoside,english,brazilian]{book}

\usepackage{packages/imegoodies}
\usepackage[thesis]{packages/imelooks}

\graphicspath{{fig/},{logos/},{img/},{images/},{imagens/}}

% Comandos rápidos para mudar de língua:
% \en -> muda para o inglês
% \br -> muda para o português
% \texten{blah} -> o texto "blah" é em inglês
% \textbr{blah} -> o texto "blah" é em português
\babeltags{br = brazilian, en = english}


%%%%%%%%%%%%%%%%%%%%%%%%%%%%%%%%%%%%%%%%%%%%%%%%%%%%%%%%%%%%%%%%%%%%%%%%%%%%%%%%
%%%%%%%%%%%%%%%%%%%%%%%%%%%%%%%%%% METADADOS %%%%%%%%%%%%%%%%%%%%%%%%%%%%%%%%%%%
%%%%%%%%%%%%%%%%%%%%%%%%%%%%%%%%%%%%%%%%%%%%%%%%%%%%%%%%%%%%%%%%%%%%%%%%%%%%%%%%

% O arquivo com os dados bibliográficos para biblatex; você pode usar
% este comando mais de uma vez para acrescentar múltiplos arquivos
\addbibresource{bibliografia.bib}

% Este comando permite acrescentar itens à lista de referências sem incluir
% uma referência de fato no texto (pode ser usado em qualquer lugar do texto)
%\nocite{bronevetsky02,schmidt03:MSc, FSF:GNU-GPL, CORBA:spec, MenaChalco08}
% Com este comando, todos os itens do arquivo .bib são incluídos na lista
% de referências
%\nocite{*}

% É possível definir como determinadas palavras podem (ou não) ser
% hifenizadas; no entanto, a hifenização automática geralmente funciona bem
\babelhyphenation{documentclass latexmk soft-ware clsguide} % todas as línguas
\babelhyphenation[brazilian]{Fu-la-no}
\babelhyphenation[english]{what-ever}

\title{Algebraic algorithm for maximum matching}[Algebraic algorithm for maximum matching]
\translatedtitle{Emparelhamento máximo}[Algoritmo algébrico para emparelhamento máximo]

\author{Antonio Marcos Shiro Arnauts Hachisuca}

\orientador{Prof. Dr. Marcel Kenji de Carli Silva}

\banca{
  \profa{} \dra{} Fulana de Tal (orientadora) -- IME-USP [sem ponto final],
  % Em inglês, não há o "ª"
  %Prof. Dr. Fulana de Tal (advisor) -- IME-USP [sem ponto final],
  Prof. Dr. Ciclano de Tal -- IME-USP [sem ponto final],
  \profa{} \dra{} Convidada de Tal -- IMPA [sem ponto final],
}

% A página de rosto da versão para depósito (ou seja, a versão final
% antes da defesa) deve ser diferente da página de rosto da versão
% definitiva (ou seja, a versão final após a incorporação das sugestões
% da banca).
\tipotese{
  tcc,
  programa={Ciência da Computação},
}

\defesa{
  local={São Paulo},
  data=2024-01-01, % YYYY-MM-DD
}

% A licença do seu trabalho. Use CC-BY, CC-BY-NC, CC-BY-ND, CC-BY-SA,
% CC-BY-NC-SA ou CC-BY-NC-ND para escolher a licença Creative Commons
% correspondente (o sistema insere automaticamente o texto da licença).
% Se quiser estabelecer regras diferentes para o uso de seu trabalho,
% converse com seu orientador e coloque o texto da licença aqui, mas
% observe que apenas TCCs sob alguma licença Creative Commons serão
% acrescentados ao BDTA. Se você tem alguma intenção de publicar o
% trabalho comercialmente no futuro, sugerimos a licença CC-BY-NC-ND.
%
%\direitos{CC-BY-NC-ND}
%
%\direitos{Autorizo a reprodução e divulgação total ou parcial deste
%          trabalho, por qualquer meio convencional ou eletrônico,
%          para fins de estudo e pesquisa, desde que citada a fonte.}
%
%\direitos{I authorize the complete or partial reproduction and disclosure
%          of this work by any conventional or electronic means for study
%          and research purposes, provided that the source is acknowledged.}
%
\direitos{CC-BY}

% Para gerar a ficha catalográfica, acesse https://fc.ime.usp.br/,
% preencha o formulário e escolha a opção "Gerar Código LaTeX".
% Basta copiar e colar o resultado aqui.
\fichacatalografica{}

% Configurações para teoremas, definições, etc.
\usepackage{mdframed}

\setlist{itemsep=0.5pt, parsep=0pt, topsep=0pt, partopsep=0pt}

%%%%%%%%%%%%%% Definitions %%%%%%%%%%%%%%%%%%%%%%
\newcommand{\Naturals}{\mathbb{N}}
\newcommand{\Integers}{\mathbb{Z}}
\newcommand{\Reals}{\mathbb{R}}
\newcommand*{\symdiff}{\mathbin{\Delta}}

\newcommand{\SC}[1]{\textsc{#1}}

\newmdenv[linecolor=black,linewidth=1pt]{problembox}

\newenvironment{problem}[1]
{
    \begin{problembox}
    \begin{center}
        \Large\SC{#1}
    \end{center}
	\itshape
	\noindent
}
{
    \end{problembox}
}

\newcommand\numberthis{\addtocounter{equation}{1}\tag{\theequation}}

%%%%%%%%%%%%% Theorems, lemmas, etc. %%%%%%%%%%%%

% Hints
% \newtheoremstyle{maybe} % nome do estilo
% {3pt} % espaço antes
% {3pt} % espaço depois
% {\itshape} % fonte do corpo
% {} % Indentação
% {\bfseries} % fonte do título
% {:} % pontuação após o título
% {.5em} % espaço após o título
% {\thmname{#1}\space(?)\thmnumber{ #2}\thmnote{ (#3)}} % formato do título
% % vazio significa {\thmname{#1}\thmnumber{ #2}\thmnote{ (#3)}}
\newtheorem{theorem}{Theorem}[section]

% Lemma style
\newtheorem{lemma}[theorem]{Lemma}

%% Definition style
\newtheorem{definition}[theorem]{Definition}

% Corollary style
\newtheorem{corollary}[theorem]{Corollary}

% Fact style
\newtheorem{fact}[theorem]{Fact}
\newtheorem*{fact*}{Fact}

% Proof style
\makeatletter
\renewenvironment{proof}[1][\proofname]{\par
  \pushQED{\qed}%
  \normalfont \topsep0pt \partopsep0pt % Removes extra vertical space before proof
  \trivlist
  \item[\hskip\labelsep
        \itshape
    #1\@addpunct{.}]\ignorespaces
}{%
  \popQED\endtrivlist\@endpefalse
}
\makeatother

\BeforeBeginEnvironment{theorem}{\vspace{0.4\baselineskip}}
\BeforeBeginEnvironment{lemma}{\vspace{0.4\baselineskip}}
\BeforeBeginEnvironment{fact}{\vspace{0.4\baselineskip}}
\BeforeBeginEnvironment{proposition}{\vspace{0.4\baselineskip}}
\BeforeBeginEnvironment{corollary}{\vspace{0.4\baselineskip}}
\BeforeBeginEnvironment{definition}{\vspace{0.4\baselineskip}}
\BeforeBeginEnvironment{example}{\vspace{0.4\baselineskip}}
\BeforeBeginEnvironment{remark}{\vspace{0.4\baselineskip}}

\begin{document}

%%%%%%%%%%%%%%%%%%%%%%%%%%% CAPA E PÁGINAS INICIAIS %%%%%%%%%%%%%%%%%%%%%%%%%%%%

% Aqui começa o conteúdo inicial que aparece antes do capítulo 1, ou seja,
% página de rosto, resumo, sumário etc. O comando frontmatter faz números
% de página aparecem em algarismos romanos ao invés de arábicos e
% desabilita a contagem de capítulos.
\frontmatter

\pagestyle{plain}

\onehalfspacing % Espaçamento 1,5 na capa e páginas iniciais

\maketitle % capa e folha de rosto

%%%%%%%%%%%%%%%% DEDICATÓRIA, AGRADECIMENTOS, RESUMO/ABSTRACT %%%%%%%%%%%%%%%%%%

\begin{dedicatoria}
Esta seção é opcional e fica numa página separada; ela pode ser usada para
uma dedicatória ou epígrafe.
\end{dedicatoria}

% Reinicia o contador de páginas (a próxima página recebe o número "i") para
% que a página da dedicatória não seja contada.
\pagenumbering{roman}

% Agradecimentos:
% Se o candidato não quer fazer agradecimentos, deve simplesmente eliminar
% esta página. A epígrafe, obviamente, é opcional; é possível colocar
% epígrafes em todos os capítulos. O comando "\chapter*" faz esta seção
% não ser incluída no sumário.
\chapter*{Agradecimentos}
\epigrafe{Do. Or do not. There is no try.}{Mestre Yoda}

Texto texto texto texto texto texto texto texto texto texto texto texto texto
texto texto texto texto texto texto texto texto texto texto texto texto texto
texto texto texto texto texto texto texto texto texto texto texto texto texto
texto texto texto texto. Texto opcional.

%%%%%%%%%%%%%%%%%%%%%%%%%%% LISTAS DE FIGURAS ETC. %%%%%%%%%%%%%%%%%%%%%%%%%%%%%

% Como as listas que se seguem podem não incluir uma quebra de página
% obrigatória, inserimos uma quebra manualmente aqui.
\cleardoublepage

\newcommand\disablenewpage[1]{{\let\clearpage\par\let\cleardoublepage\par #1}}

% Nestas listas, é melhor usar "raggedbottom" (veja basics.tex). Colocamos
% a opção correspondente e as listas dentro de um grupo para ativar
% raggedbottom apenas temporariamente.
\bgroup
\raggedbottom

%%%%% Listas criadas manualmente

%\chapter*{Lista de abreviaturas}
\disablenewpage{\chapter*{Lista de abreviaturas}}

\begin{tabular}{rl}
%  ABNT & Associação Brasileira de Normas Técnicas\\
   URL & Localizador Uniforme de Recursos (\emph{Uniform Resource Locator})\\
   IME & Instituto de Matemática e Estatística\\
   USP & Universidade de São Paulo
\end{tabular}

%\chapter*{Lista de símbolos}
\disablenewpage{\chapter*{Lista de símbolos}}

\begin{tabular}{rl}
%  $\omega$ & Frequência angular\\
%    $\psi$ & Função de análise \emph{wavelet}\\
%    $\Psi$ & Transformada de Fourier de $\psi$\\
\end{tabular}

% Quebra de página manual
\clearpage

%%%%% Listas criadas automaticamente

% Você pode escolher se quer ou não permitir a quebra de página
%\listoffigures
\disablenewpage{\listoffigures}

% Você pode escolher se quer ou não permitir a quebra de página
%\listoftables
\disablenewpage{\listoftables}

\disablenewpage{\listof{program}{\programlistname}}

% Sumário (obrigatório)
\tableofcontents

\egroup % Final de "raggedbottom"

% Referências indiretas ("x", veja "y") para o índice remissivo (opcionais,
% pois o índice é opcional). É comum colocar esses itens no final do documento,
% junto com o comando \printindex, mas em alguns casos isso torna necessário
% executar texindy (ou makeindex) mais de uma vez, então colocar aqui é melhor.
\index{Inglês|see{Língua estrangeira}}
\index{Figuras|see{Floats}}
\index{Tabelas|see{Floats}}
\index{Código-fonte|see{Floats}}
\index{Subcaptions|see{Subfiguras}}
\index{Sublegendas|see{Subfiguras}}
\index{Equações|see{Modo matemático}}
\index{Fórmulas|see{Modo matemático}}
\index{Rodapé, notas|see{Notas de rodapé}}
\index{Captions|see{Legendas}}
\index{Versão original|see{Tese/Dissertação, versões}}
\index{Versão corrigida|see{Tese/Dissertação, versões}}
\index{Palavras estrangeiras|see{Língua estrangeira}}
\index{Floats!Algoritmo|see{Floats, ordem}}


%%%%%%%%%%%%%%%%%%%%%%%%%%%%%%%% CAPÍTULOS %%%%%%%%%%%%%%%%%%%%%%%%%%%%%%%%%%%%%

% Aqui vai o conteúdo principal do trabalho, ou seja, os capítulos que compõem
% a dissertação/tese. O comando mainmatter reinicia a contagem de páginas,
% modifica a numeração para números arábicos e ativa a contagem de capítulos.
\mainmatter

\pagestyle{mainmatter}

% Espaçamento simples
\singlespacing

% A introdução não tem número de capítulo, então os cabeçalhos também não
\pagestyle{unnumberedchapter}
\chapter**{Introduction}
\label{cap:introduction}

\enlargethispage{.5\baselineskip}

Something.

\pagestyle{mainmatter}
\renewcommand*{\proofname}{Proof}

\chapter{Preliminaries}

The purpose of this chapter is to introduce key concepts related to the maximum matching algorithm. 
The chapter covers important topics such as the definition of graph maximum matching, the Sherman-Morrison-Woodbury formula, and the Schur complement. 
These concepts are fundamental for understanding the algorithm's correctness and time complexity.

\enlargethispage{.5\baselineskip}

\section{Graph theory}
\label{sec:graph}

\begin{definition}[Graph]
	\label{def:graph}
	A \textbf{graph} \(G\) is a triple \((V, E, \varphi)\) such that
	\begin{enumerate}[label=(\roman*)]
		\item \(V\) is the \textbf{vertex set};
		\item \(E\) is the \textbf{edge set};
		\item \(\varphi: E \to V \times V\) is a relation between each edge and a pair of vertices, called the \textbf{incidence function} of \(G\).
	\end{enumerate}
	Usually, it is used 
	\(V(G)\) or \(V_G\) to denote \(V\) and 
	\(E(G)\) or \(E_G\) to denote \(E\).
	Also, if \(e \in E(G)\) and \(\varphi(e) = (u, v)\), then \(u\) and \(v\) are the \textbf{ends} of \(e\);
	When the context is clear, \((u, v)\) may be abbreviated to \(uv\).
\end{definition}

\begin{definition}[Matching]
	\label{def:matching}
	For a graph \(G \coloneqq (V, E, \varphi)\), a set \(M \subseteq E\) is a \textbf{matching} of \(G\) if and only if no two edges in \(M\) share an end.
	A vertex \(v \in V\) is \(M\)-covered if some edge of \(M\) incides in \(v\), 
	and it is said that \(M\) covers \(v\);
	Otherwise, \(v\) is \(M\)-exposed.
	A matching \(M\) is:
	\begin{itemize}
		\item 
			\textbf{maximal}, if there is no edge \(e \in E \setminus M\) such that \(M \cup \{e\}\) is a matching of \(G\);

		\item
			\textbf{maximum}, if for every matching \(M'\) of \(G\) one has \(|M| \geq |M'|\);
	
		\item
			\textbf{perfect}, if \(2|V_G| = |M|\), i.e., every vertex of \(G\) is covered.
	\end{itemize}
	Now, the following problem can be introduced.
	\\
\end{definition}
\begin{problem}{Maximum matching}
	Given a graph, find a maximum matching of this graph.
\end{problem}
\section{Linear algebra}
\label{sec:linear_algebra}

% \begin{definition}[Fields]
% \label{def:fields}
% A \textbf{field} is a set with two binary operations on \(\mathbb{F}\) called \textit{addition} and \textit{multiplication}.
% The addition of two elements \(a\) and \(b\) from \(\mathbb{F}\) is denoted as \(a + b\).
% The multiplication of two elements \(a\) and \(b\) from \(\mathbb{F}\) is denoted as \(a \cdot b\).
% These operations must satisfy the following properties:
% \begin{enumerate}
%     \item Associativity: \(a + (b + c) = (a + b) + c\) and \(a \cdot (b \cdot c) = (a \cdot b) \cdot c\);
%     \item Commutativity: \(a + b = b + a\) and \(a \cdot b = b \cdot a\);
%     \item Additive and multiplicative identity: there exists two elements \(0\) and \(1\) in \(\mathbb{F}\) such that \(a + 0 = a\) and \(a \cdot 1 = a\);
%     \item Additive inverse: there exists an element \(-a\) such that \(a + (-a) = 0\);
%     \item Multiplicative inverse: there exists an element \(a^{-1}\) such that \(a \cdot a^{-1} = 1\);
%     \item Distributivity: 
% \end{enumerate}
% \end{definition}

\begin{definition}[Submatrix]
    Let \(M\) be a matrix, we say that \(M'\) is a \textbf{submatrix} of \(M\) if we can obtain \(M'\) by removing zero or more rows and/or columns from \(M\).
\end{definition}
\noindent
Let \(M\) be a matrix. 
For any sets of indices \(R\) and \(C\), we write \(M_{R,C}\) or \(M[R,C]\) to denote the submatrix of \(M\) formed by keeping only the rows indexed by \(R\) and columns indexed by \(C\). 
Furthermore, we use \(M[R,*]\) or \(M_{R, *}\) to represent the submatrix containing all rows indexed by \(R\) and all columns of \(M\) (resp., \(M[*, C]\)).
% \begin{definition}[Matrix inverse]
%     Let \(M\) be a matrix. 
%     The \textbf{inverse} of \(M\), denoted as \(M^{-1}\), is a matrix such that \(MM^{-1} = I\)
% \end{definition}
% 
% \begin{definition}[Singular matrix]
%     A matrix \(M\) is \textbf{singular} if it does not have an inverse.
%     A matrix is \textbf{non-singular} if it is not singular.
% \end{definition}

\begin{definition}[Schur complement]
    \label{def:schur}
    Let \(M\) be a square matrix of the form
    \[
        M =
        \begin{pmatrix}
            A & B \\
            C & D
        \end{pmatrix}
    \]
    where \(D\) is a square matrix; Then, if \(D\) is non-singular, the matrix \(A - B D^{-1} C\) is the \textbf{Schur complement} of \(D\) in \(M\).
    The Schur complement has the following properties:
    \begin{enumerate}
      \item \(\det(M) = \det(D)\det(A - BD^{-1}C)\).
    \end{enumerate}
\end{definition}

\begin{theorem}[Sherman-Morrison-Woodbury formula]
\label{thm:smw-formula}
  Suppose that \(M\) is non-singular. 
  Then
  \begin{enumerate}[label = (\arabic*)]
      \item \(M + UV^T\) is non-singular if and only if \(I + V^TM^{-1}U\) is non-singular;
      \item If \(M + UV^T\) is non-singular, then
      \[
          (M + UV^T)^{-1} = M^{-1} - M^{-1}U(I + V^TM^{-1}U)^{-1}V^TM^{-1}.
      \]
  \end{enumerate}
\end{theorem}

\begin{proof}
    For (1), let 
    \[
      A \coloneqq \begin{bmatrix} I & V^T \\ -U & M \end{bmatrix}.
    \]
    Note that the Schur complement of the block \(M\) of the matrix \(A\) is \(I + V^TM^{-1}U\) and that 
    \[
      \begin{bmatrix} I & V^T \\ -U & M \end{bmatrix} = 
      \begin{bmatrix} I & 0 \\ -U & I \end{bmatrix} \begin{bmatrix} I & V^T \\ 0 & M + UV^T \end{bmatrix}.
    \]
    Then, one has
    \begin{align*}
      \det(M)\det(I + V^TM^{-1}U) 
      &= \det(A) \\
      &= \det\bigg(\begin{bmatrix} I & 0 \\ -U & I \end{bmatrix} \begin{bmatrix} I & V^T \\ 0 & M + UV^T \end{bmatrix}\bigg) \\
      &= \det(I) \det(M+UV^T) = \det(M+UV^T).
    \end{align*}
    Thus, (1) is proven.
    For (2), it suffices to verify that 
    \[
        (M + UV^T)(M^{-1} - M^{-1}U(I + V^TM^{-1}U)^{-1}V^TM^{-1}) = I.
    \]
    Let \(A \coloneqq (I + V^TM^{-1}U)\), then
    \begin{align*}
        &\!\!\!\!\!\!\!\!\!\!\!\!\!\!\!\!\!\! (M + UV^T)(M^{-1} - M^{-1}U(I + V^TM^{-1}U)^{-1}V^TM^{-1})\\
        = \, &(M + UV^T)(M^{-1} - M^{-1}UA^{-1}V^TM^{-1})\\
        = \, &(I - UA^{-1}V^TM^{-1}) + (UV^TM^{-1} - UV^TM^{-1}UA^{-1}V^TM^{-1}) \\
        = \, &(I + UV^TM^{-1}) - (UA^{-1}V^TM^{-1} + UV^TM^{-1}UA^{-1}V^TM^{-1}) \\
        = \, &(I + UV^TM^{-1}) - U(I + V^TM^{-1}U)(A^{-1}V^TM^{-1}) \\
        = \, &(I + UV^TM^{-1}) - UAA^{-1}V^TM^{-1} \\
        = \, &I + UV^TM^{-1} - UV^TM^{-1} = I. \qedhere \\
    \end{align*}
\end{proof}

\begin{corollary}[{\citet[Corollary 2.1]{Harvey:Paper}}]
    \label{cor:update_cor} 
    Let \(M\) be a non-singular square matrix, let \(N \coloneqq M^{-1}\) and let \(S\) be a subset of rows from \(M\).
    Let \(\tilde{M}\) be a matrix that which is identical to \(M\) except that \(\tilde{M}_{S, S} \neq M_{S, S}\)
    and \(\Delta \coloneqq \tilde{M}_{S, S} - M_{S, S}\).
    If \(\tilde{M}\) is non-singular, then
    \[
        \tilde{M}^{-1} = N - N_{*, S}(I + \Delta N_{S, S})^{-1}\Delta N_{S, *}.
    \]
\end{corollary}

\begin{proof}
    Let \(k \coloneqq |S|\). 
    Let \(U\)\footnote{Every column should be a canonical basis vector from \(S\)} be a \(n \times k\) matrix that selects the \(S\) columns from a \(n \times n\) matrix, i.e.
    for a \(n \times n\) matrix \(M\), one has \(MU = M_{*, S}\). Let \(V^T \coloneqq \Delta U^T\). 
    Then, by \cref{thm:smw-formula}, 
    \begin{align*}
        \tilde{M}^{-1} &= (M + U V^T)^{-1} \\
        &= N - N U (I + \Delta U^T N U)^{-1} \Delta U^T N \\
        &= N - N_{*, S} (I + \Delta U^T N U)^{-1} \Delta N_{S, *} \\
        &= N - N_{*, S} (I + \Delta N_{S, S})^{-1} \Delta N_{S, *}. \qedhere
    \end{align*}
\end{proof}

\begin{definition}[Skew-symmetric matrix]
\label{def:skew}
    A matrix \(M\) is \textbf{skew-symmetric} if \(M = -M^{T}\).
\end{definition}

\begin{fact}[Inverse of a skew-symmetric matrix]
    Let \(M\) be a skew-symmetric matrix.
    If \(M\) is non-singular, then \(M^{-1}\) is also skew-symmetric.
\end{fact}

\section{Matrix Algorithms}
\label{matrix:time_complexity}

% This section presents the matrix algorithms used throughout this paper.

\begin{definition}[Matrix multiplication exponent]
  The \textbf{matrix exponent multiplication exponent}, denoted by \(\omega\), is the infimum of the exponent over all matrix multiplication algorithms.
  That is there is a known algorithm that solves matrix multiplication in \(O(n^{\omega + o(1)})\) field operations.
\end{definition}

\noindent
The following milestones related to matrix multiplication algorithms were found:
\begin{center}
  \begin{tabular}{|c|c|c|}
    \hline
    Year & Author(s) & Bound on omega \\
    \hline
    1969 & \citet{Strassen1969} & 2.8074 \\ 
    1990 & \citet{COPPERSMITH1990} & 2.3755 \\
    2024 & \citet{2024asymmetryyieldsfastermatrix} & 2.371339 \\
    \hline
  \end{tabular}
\end{center}

\noindent
The following algorithms can be done in \(O(n^\omega)\):

\begin{enumerate}
  \item \textbf{Matrix inversion}, see \citet[Theorem 28.2]{CLRS}; 
  \item \textbf{Matrix rank}, see (TODO: add citation).
\end{enumerate}


\chapter{Simpler(?) algorithms}

\begin{definition}[Indeterminates]

\end{definition}

\begin{definition}[Tutte Matrix]
    Given a graph \(G\) and a function that maps \(f\) every edge of \(G\) to an indeterminate.
    Then, a tutte matrix is a matrix such that, for each \(uv \in E(G)\), one has \(T_{uv} = -T_{vu} = f(vu)\).
\end{definition}

\begin{fact}[Tutte matrix perfect matching condition]
    \label{fact:matching_condition}
    A graph \(G\) has a perfect matching iff \(T_G\) is non-singular.
\end{fact}

\begin{proof}
    This is direct from Pffafian.
\end{proof}

\section{Basic algorithm}

By fact \ref{fact:matching_condition}, one can achieve a very direct algorithm.
The idea is try to remove an edge \(e\), if \(G - e\) has a perfect matching, 
then this edge can be removed; Else, \(e\) belongs to a perfect matching of \(G\).
Repeat this step until only the perfect matching edges are left.
Such approach achieves an \(O(n^{\omega + 2})\) time complexity.

\section{Rank-2 update algorithm}

One of the bottlenecks of the previous algorithm is the necessity to check after each
edge if the matrix is non-singular, each of these checks is \(O(n^{\omega})\).
Thus, one can maintain the inverse through rank-2 updates.
Achieving a time complexity of \(O(n^{4})\).
For each edge, it suffices to check if \(N_{ij} \neq -1/T_{ij}\).
This condition is direct from corollary \ref{cor:1}.

\chapter{Harvey's algorithm}

\begin{enumerate}
    \item Brief of the idea;
    \item Pseudo-algorithm;
    \item Corretude;
\end{enumerate}

\chapter{Extension to Maximum Matching}

\begin{enumerate}
    \item Theorem of Lovasz, the size of a maximum matching is the rank of the matrix;
    \item Prove this theorem;
    \item Extend graph to have a perfect matching and remove added vertices.
\end{enumerate}

%%%%%%%%%%%%%%% SEÇÕES FINAIS (BIBLIOGRAFIA E ÍNDICE REMISSIVO) %%%%%%%%%%%%%%%%

% O comando backmatter desabilita a numeração de capítulos.
\backmatter

\pagestyle{backmatter}

% Espaço adicional no sumário antes das referências / índice remissivo
\addtocontents{toc}{\vspace{2\baselineskip plus .5\baselineskip minus .5\baselineskip}}

% A bibliografia é obrigatória

\printbibliography[
  title=\refname\label{sec:bib}, % "Referências", recomendado pela ABNT
  %title=\bibname\label{sec:bib}, % "Bibliografia"
  heading=bibintoc, % Inclui a bibliografia no sumário
]

\printindex % imprime o índice remissivo no documento (opcional)

\end{document}
