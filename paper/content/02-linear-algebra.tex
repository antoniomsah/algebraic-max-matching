\section{Linear algebra}
\label{sec:linear_algebra}

% \begin{definition}[Fields]
% \label{def:fields}
% A \textbf{field} is a set with two binary operations on \(\mathbb{F}\) called \textit{addition} and \textit{multiplication}.
% The addition of two elements \(a\) and \(b\) from \(\mathbb{F}\) is denoted as \(a + b\).
% The multiplication of two elements \(a\) and \(b\) from \(\mathbb{F}\) is denoted as \(a \cdot b\).
% These operations must satisfy the following properties:
% \begin{enumerate}
%     \item Associativity: \(a + (b + c) = (a + b) + c\) and \(a \cdot (b \cdot c) = (a \cdot b) \cdot c\);
%     \item Commutativity: \(a + b = b + a\) and \(a \cdot b = b \cdot a\);
%     \item Additive and multiplicative identity: there exists two elements \(0\) and \(1\) in \(\mathbb{F}\) such that \(a + 0 = a\) and \(a \cdot 1 = a\);
%     \item Additive inverse: there exists an element \(-a\) such that \(a + (-a) = 0\);
%     \item Multiplicative inverse: there exists an element \(a^{-1}\) such that \(a \cdot a^{-1} = 1\);
%     \item Distributivity: 
% \end{enumerate}
% \end{definition}

\begin{definition}[Submatrix]
    Let \(M\) be a matrix, we say that \(M'\) is a \textbf{submatrix} of \(M\) if we can obtain \(M'\) by removing zero or more rows and/or columns from \(M\).
\end{definition}
\noindent
Let \(M\) be a matrix. 
For any sets of indices \(R\) and \(C\), we write \(M_{R,C}\) or \(M[R,C]\) to denote the submatrix of \(M\) formed by keeping only the rows indexed by \(R\) and columns indexed by \(C\). 
Furthermore, we use \(M[R,*]\) or \(M_{R, *}\) to represent the submatrix containing all rows indexed by \(R\) and all columns of \(M\) (resp., \(M[*, C]\)).
% \begin{definition}[Matrix inverse]
%     Let \(M\) be a matrix. 
%     The \textbf{inverse} of \(M\), denoted as \(M^{-1}\), is a matrix such that \(MM^{-1} = I\)
% \end{definition}
% 
% \begin{definition}[Singular matrix]
%     A matrix \(M\) is \textbf{singular} if it does not have an inverse.
%     A matrix is \textbf{non-singular} if it is not singular.
% \end{definition}

\begin{definition}[Schur complement]
    \label{def:schur}
    Let \(M\) be a square matrix of the form
    \[
        M =
        \begin{pmatrix}
            A & B \\
            C & D
        \end{pmatrix}
    \]
    where \(D\) is a square matrix; Then, if \(D\) is non-singular, the matrix \(A - B D^{-1} C\) is the \textbf{Schur complement} of \(D\) in \(M\).
    The Schur complement has the following properties:
    \begin{enumerate}
      \item \(\det(M) = \det(D)\det(A - BD^{-1}C)\).
    \end{enumerate}
\end{definition}

\begin{theorem}[Sherman-Morrison-Woodbury formula]
\label{thm:smw-formula}
  Suppose that \(M\) is non-singular. 
  Then
  \begin{enumerate}[label = (\arabic*)]
      \item \(M + UV^T\) is non-singular if and only if \(I + V^TM^{-1}U\) is non-singular;
      \item If \(M + UV^T\) is non-singular, then
      \[
          (M + UV^T)^{-1} = M^{-1} - M^{-1}U(I + V^TM^{-1}U)^{-1}V^TM^{-1}.
      \]
  \end{enumerate}
\end{theorem}

\begin{proof}
    For (1), let 
    \[
      A \coloneqq \begin{bmatrix} I & V^T \\ -U & M \end{bmatrix}.
    \]
    Note that the Schur complement of the block \(M\) of the matrix \(A\) is \(I + V^TM^{-1}U\) and that 
    \[
      \begin{bmatrix} I & V^T \\ -U & M \end{bmatrix} = 
      \begin{bmatrix} I & 0 \\ -U & I \end{bmatrix} \begin{bmatrix} I & V^T \\ 0 & M + UV^T \end{bmatrix}.
    \]
    Then, one has
    \begin{align*}
      \det(M)\det(I + V^TM^{-1}U) 
      &= \det(A) \\
      &= \det\bigg(\begin{bmatrix} I & 0 \\ -U & I \end{bmatrix} \begin{bmatrix} I & V^T \\ 0 & M + UV^T \end{bmatrix}\bigg) \\
      &= \det(I) \det(M+UV^T) = \det(M+UV^T).
    \end{align*}
    Thus, (1) is proven.
    For (2), it suffices to verify that 
    \[
        (M + UV^T)(M^{-1} - M^{-1}U(I + V^TM^{-1}U)^{-1}V^TM^{-1}) = I.
    \]
    Let \(A \coloneqq (I + V^TM^{-1}U)\), then
    \begin{align*}
        &\!\!\!\!\!\!\!\!\!\!\!\!\!\!\!\!\!\! (M + UV^T)(M^{-1} - M^{-1}U(I + V^TM^{-1}U)^{-1}V^TM^{-1})\\
        = \, &(M + UV^T)(M^{-1} - M^{-1}UA^{-1}V^TM^{-1})\\
        = \, &(I - UA^{-1}V^TM^{-1}) + (UV^TM^{-1} - UV^TM^{-1}UA^{-1}V^TM^{-1}) \\
        = \, &(I + UV^TM^{-1}) - (UA^{-1}V^TM^{-1} + UV^TM^{-1}UA^{-1}V^TM^{-1}) \\
        = \, &(I + UV^TM^{-1}) - U(I + V^TM^{-1}U)(A^{-1}V^TM^{-1}) \\
        = \, &(I + UV^TM^{-1}) - UAA^{-1}V^TM^{-1} \\
        = \, &I + UV^TM^{-1} - UV^TM^{-1} = I. \qedhere \\
    \end{align*}
\end{proof}

\begin{corollary}[{\citet[Corollary 2.1]{Harvey:Paper}}]
    \label{cor:update_cor} 
    Let \(M\) be a non-singular square matrix, let \(N \coloneqq M^{-1}\) and let \(S\) be a subset of rows from \(M\).
    Let \(\tilde{M}\) be a matrix that which is identical to \(M\) except that \(\tilde{M}_{S, S} \neq M_{S, S}\)
    and \(\Delta \coloneqq \tilde{M}_{S, S} - M_{S, S}\).
    If \(\tilde{M}\) is non-singular, then
    \[
        \tilde{M}^{-1} = N - N_{*, S}(I + \Delta N_{S, S})^{-1}\Delta N_{S, *}.
    \]
\end{corollary}

\begin{proof}
    Let \(k \coloneqq |S|\). 
    Let \(U\)\footnote{Every column should be a canonical basis vector from \(S\)} be a \(n \times k\) matrix that selects the \(S\) columns from a \(n \times n\) matrix, i.e.
    for a \(n \times n\) matrix \(M\), one has \(MU = M_{*, S}\). Let \(V^T \coloneqq \Delta U^T\). 
    Then, by \cref{thm:smw-formula}, 
    \begin{align*}
        \tilde{M}^{-1} &= (M + U V^T)^{-1} \\
        &= N - N U (I + \Delta U^T N U)^{-1} \Delta U^T N \\
        &= N - N_{*, S} (I + \Delta U^T N U)^{-1} \Delta N_{S, *} \\
        &= N - N_{*, S} (I + \Delta N_{S, S})^{-1} \Delta N_{S, *}. \qedhere
    \end{align*}
\end{proof}

\begin{definition}[Skew-symmetric matrix]
\label{def:skew}
    A matrix \(M\) is \textbf{skew-symmetric} if \(M = -M^{T}\).
\end{definition}

\begin{fact}[Inverse of a skew-symmetric matrix]
    Let \(M\) be a skew-symmetric matrix.
    If \(M\) is non-singular, then \(M^{-1}\) is also skew-symmetric.
\end{fact}

\section{Matrix Algorithms}
\label{matrix:time_complexity}

% This section presents the matrix algorithms used throughout this paper.

\begin{definition}[Matrix multiplication exponent]
  The \textbf{matrix exponent multiplication exponent}, denoted by \(\omega\), is the infimum of the exponent over all matrix multiplication algorithms.
  That is there is a known algorithm that solves matrix multiplication in \(O(n^{\omega + o(1)})\) field operations.
\end{definition}

\noindent
The following milestones related to matrix multiplication algorithms were found:
\begin{center}
  \begin{tabular}{|c|c|c|}
    \hline
    Year & Author(s) & Bound on omega \\
    \hline
    1969 & \citet{Strassen1969} & 2.8074 \\ 
    1990 & \citet{COPPERSMITH1990} & 2.3755 \\
    2024 & \citet{2024asymmetryyieldsfastermatrix} & 2.371339 \\
    \hline
  \end{tabular}
\end{center}

\noindent
The following algorithms can be done in \(O(n^\omega)\):

\begin{enumerate}
  \item \textbf{Matrix inversion}, see \citet[Theorem 28.2]{CLRS}; 
  \item \textbf{Matrix rank}, see (TODO: add citation).
\end{enumerate}

