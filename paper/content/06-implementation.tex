\chapter{Implementation}

\section{Matrix implementation}
\label{sec:matrix_impl}

In this section, fundamental concepts related to matrix implementation are presented such as time complexity of various matrix operations,
matrix multiplication being as hard to compute as matrix inversion and others.
The objective is providing a sufficient understanding of key matrix operations.

\subsection{Definitions}

\begin{definition}[Schur complement]
    \label{def:schur}
    Let \(M\) be a square matrix of the form
    \[
        M = 
        \begin{pmatrix}
            W & X \\
            Y & Z 
        \end{pmatrix}
    \]
    where \(Z\) is a square matrix; Then, if \(Z\) is non-singular, the matrix
    \[
        C = W - X Z^{-1} Y
    \]
    is the \textit{Schur complement} of \(Z\) in \(M\).
\end{definition}

\subsection{Matrix operations}

\subsection{Matrix inverse}


An \(O(n^\omega)\) algorithm that computes the matrix inverse can be achieved exclusively using matrix multiplication with complexity \(O(n^\omega)\). 
The theorem and pseudo-code below demonstrates the process.

\begin{theorem}[Matrix inversions is no harder than matrix multiplication]
    \label{thm:mult_inv}
    If there is an algorithm that computes matrix multiplication in \(O(n^{\omega})\), then
    there is an algorithm that computes matrix inverse in \(O(n^{\omega})\).
\end{theorem}

\begin{proof}
    \[
        A^{-1} = 
        \begin{pmatrix}
            B^{-1} + B^{-1}C^TS^{-1}CB^{-1} & -B^{-1}C^TS^{-1} \\
            -S^{-1}CB^{-1} & S^{-1}
        \end{pmatrix}
    \]
\end{proof}

Now, the following algorithm can be implemented.
\begin{programruledcaption}{Matrix: \(\SC{Inverse}\)}
    \noindent
    \SC{Input:} A \(n \times n\) matrix \(A\) where \(n\) is a power of two; \\
    \SC{Output: } \(A^{-1}\). 

    \noindent \hrule
    \begin{lstlisting}[
      language={pseudocode},
      style=pseudocode,
      style=wider,
      functions={MatrixInverse},
      specialidentifiers={},
    ]
    function MatrixInverse(A) 
        if n = 1  
            if $A_{0, 0} = 0$ 
                // Matrix is singular.
            end
            $N_{0,0}$ := $1 / A_{0,0}$
            return N
        end
        $\begin{pmatrix} B & C \\ C^T & D\end{pmatrix}$ := A // Each submatrix size should be $n/2 \times n/2$.
        S := $D - CB^{-1}C^T$ // Schur complement (\ref{def:schur}).
        // Save results for $B^{-1}$ and $S^{-1}$.
        return $ \begin{pmatrix} B^{-1} + B^{-1}C^TS^{-1}CB^{-1} & -B^{-1}C^TS^{-1} \\ -S^{-1}CB^{-1} & S^{-1} \end{pmatrix}$ // \cref{thm:mult_inv}.
    end
    \end{lstlisting}
\end{programruledcaption}

\subsection*{Complexity}
Note that the algorithm above can be optimized to only calculate the recursive inversions a single time by saving them.
Suppose it is optimized.
Each call, has two recursive calls one for \(B^{-1}\) and another for \(S^{-1}\).
Suppose \(n = \max(n, m)\).
\begin{align}
    T(n) = O(n^2) + T(n/2, m/2) + O(n^2) + 2O(n^\omega) + 2T(n/2) + O(n^\omega)
\end{align}

\subsection{Matrix rank}