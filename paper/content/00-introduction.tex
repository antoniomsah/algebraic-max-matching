\chapter**{Introduction}
\label{cap:introduction}

\enlargethispage{.5\baselineskip}

The Maximum Matching problem is a fundamental problem in graph theory. Key milestones include:
\begin{itemize}
    \item In 1965, \citet{Edmonds} introduced an \(O(nm)\) algorithm that solves this problem using graph theory techniques;
    \item In 1989, \citet{Rabin1989} developed a probabilistic \(O(n^{\omega+1})\) algorithm that solved the problem through algebraic methods;
    \item In 2009, \citet{Harvey:Paper} created a probabilistic \(O(n^\omega)\) algebraic algorithm using a divide-and-conquer approach with lazy updates.
\end{itemize}
These developments show how researchers have progressively improved algorithms to solve the Maximum Matching problem, reducing computational complexity over time.

The primary objective of this paper is to implement the algorithm developed by \citet{Harvey:Paper}. 
We present a C++ implementation of the algorithm, available in the open-source repository at \href{https://github.com/antoniomsah/algebraic-max-matching/tree/main/code}{github/antoniomsah/algebraic-max-matching}.
The dissertation is structured as follows:
\begin{enumerate}
\item \cref{chap:preliminaries} covers the fundamental concepts necessary for understanding the subsequent work;
\item \cref{chap:perfect_matching} explores Tutte matrices and algebraic methods for solving perfect matching problems;
\item \cref{chap:harvey} presents a detailed explanation of Harvey's algorithm;
\item \cref{chap:maximum_matching} shows how to extend perfect matching algorithms to maximum matching scenarios.
\end{enumerate}

Throughout this paper references to the implementation will be abbreviated to the respective path in the repository.