\renewcommand*{\proofname}{Proof}

\chapter{Preliminaries}
\label{chap:preliminaries}

The purpose of this chapter is to introduce key concepts related to the maximum matching algorithm. 
The chapter covers important topics such as the definition of graph maximum matching, the Sherman-Morrison-Woodbury formula, and skew-symmetric matrices. 
These concepts are fundamental for understanding the algorithm's correctness and time complexity.

\enlargethispage{.5\baselineskip}

\section{Graph theory}
\label{sec:graph}

\begin{definition}[Graph]
\label{def:graph}
	A \textbf{graph} \(G\) is a pair \((V, E)\) such that
	\begin{enumerate}[label=(\roman*)]
		\item \(V\) is a finite set, whose elements are called \textbf{vertices};
		\item \(E\) is set of unordered pairs of vertices, whose elements are called \textbf{edges};
	\end{enumerate}
\end{definition}
\noindent
The vertex set of a graph \(G\) is denoted as \(V_G\) or \(V(G)\).
The edge set of a graph \(G\) is denoted as \(E_G\) or \(E(G)\).
If \(\{u,v\} \in E(G)\), then \(u\) and \(v\) are the \textbf{ends} of \(e\) and \(e\) \textbf{incides} in both \(u\) and \(v\);
When the context is clear, \(\{u, v\}\) may be abbreviated to \(uv\).

\begin{definition}[Matching]
\label{def:matching}
	For a graph \(G \coloneqq (V, E)\), a subset \(M \subseteq E\) is a \textbf{matching} of \(G\) if no two edges in \(M\) share an end.
	A vertex \(v \in V\) is \(M\)-covered if some edge of \(M\) incides in \(v\), 
	and it is said that \(M\) covers \(v\);
	Otherwise, \(v\) is \(M\)-exposed.
	A matching \(M\) is:
	\begin{itemize}
		\item 
			\textbf{maximal}, if there is no edge \(e \in E \setminus M\) such that \(M \cup \{e\}\) is a matching of \(G\);
		\item
			\textbf{maximum}, if for every matching \(M'\) of \(G\) one has \(|M| \geq |M'|\);
		\item
			\textbf{perfect}, if \(|V_G| = |M|\), i.e., every vertex of \(G\) is \(M\)-covered.
	\end{itemize}
	The \textbf{matching number}, denoted as \(\nu(G)\), is the size of a maximum matching of \(G\).
\end{definition}

\begin{definition}[Essential edges]
\label{def:essential}
    Let \(G \coloneqq (V, E)\) be a graph that has a perfect matching.
    An edge \(e \in E\) is \textbf{essential} if the graph \((V, E \setminus \{e\})\) does not have a perfect matching.
    An edge is \textbf{inessential} if it is not essential.
\end{definition}
\noindent
Now, the following problem can be introduced.
\newpage
\begin{problem}{Maximum matching}
\label{prob:max_matching}
	Given a graph \(G\), find a maximum matching of \(G\).
\end{problem}
