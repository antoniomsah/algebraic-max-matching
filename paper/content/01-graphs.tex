\renewcommand*{\proofname}{Proof}

\chapter{Preliminaries}

The purpose of this chapter is to introduce key concepts related to the maximum matching algorithm. 
The chapter covers important topics such as the definition of graph maximum matching, the Sherman-Morrison-Woodbury formula, and the Schur complement. 
These concepts are fundamental for understanding the algorithm's correctness and time complexity.

\enlargethispage{.5\baselineskip}

\section{Graph theory}
\label{sec:graph}

\begin{definition}[Graph]
	\label{def:graph}
	A \textbf{graph} \(G\) is a triple \((V, E, \varphi)\) such that
	\begin{enumerate}[label=(\roman*)]
		\item \(V\) is the \textbf{vertex set};
		\item \(E\) is the \textbf{edge set};
		\item \(\varphi: E \to V \times V\) is a relation between each edge and a pair of vertices, called the \textbf{incidence function} of \(G\).
	\end{enumerate}
	Usually, it is used 
	\(V(G)\) or \(V_G\) to denote \(V\) and 
	\(E(G)\) or \(E_G\) to denote \(E\).
	Also, if \(e \in E(G)\) and \(\varphi(e) = (u, v)\), then \(u\) and \(v\) are the \textbf{ends} of \(e\);
	When the context is clear, \((u, v)\) may be abbreviated to \(uv\).
\end{definition}

\begin{definition}[Matching]
	\label{def:matching}
	For a graph \(G \coloneqq (V, E, \varphi)\), a set \(M \subseteq E\) is a \textbf{matching} of \(G\) if and only if no two edges in \(M\) share an end.
	A vertex \(v \in V\) is \(M\)-covered if some edge of \(M\) incides in \(v\), 
	and it is said that \(M\) covers \(v\);
	Otherwise, \(v\) is \(M\)-exposed.
	A matching \(M\) is:
	\begin{itemize}
		\item 
			\textbf{maximal}, if there is no edge \(e \in E \setminus M\) such that \(M \cup \{e\}\) is a matching of \(G\);

		\item
			\textbf{maximum}, if for every matching \(M'\) of \(G\) one has \(|M| \geq |M'|\);
	
		\item
			\textbf{perfect}, if \(2|V_G| = |M|\), i.e., every vertex of \(G\) is covered.
	\end{itemize}
	Now, the following problem can be introduced.
	\\
\end{definition}
\begin{problem}{Maximum matching}
	Given a graph, find a maximum matching of this graph.
\end{problem}

