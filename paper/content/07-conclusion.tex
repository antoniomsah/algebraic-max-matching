\chapter{Conclusion}

In this paper, we implemented the randomized maximum matching algorithm introduced by \citet{Harvey:Paper}. 
The algorithm's approach leverages algebraic graph theory, contrasting with traditional methods like the Edmonds-Blossom \cite{Edmonds} algorithm. 
While Harvey's algorithm achieves a superior theoretical time complexity of $O(n^\omega)$ (where $n$ represents the number of vertices and $\omega$ is the matrix multiplication exponent), our implementation analysis in \cref{results:perfect_matching} and \cref{results:max_matching} reveals significant practical limitations. 
The high constant factor in the complexity makes the algorithm less efficient for smaller graphs.
Theoretically, a performance threshold exists where this approach surpasses the Edmonds-Blossom algorithm, but this threshold was neither found nor tested.

Nonetheless, when considering the implementation aspects, Harvey's algorithm offers notable advantages. 
Setting aside the mathematical foundations behind the updates (e.g. \cref{update:1}), the algorithm provides a significantly more intuitive and straightforward implementation compared to the Edmonds-Blossom algorithm. 
This simplicity makes it particularly valuable for applications involving larger graphs or scenarios where absolute computational efficiency is not the primary concern.
