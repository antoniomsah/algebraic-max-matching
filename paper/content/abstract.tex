%!TeX root=../tese.tex
%("dica" para o editor de texto: este arquivo é parte de um documento maior)
% para saber mais: https://tex.stackexchange.com/q/78101

% As palavras-chave são obrigatórias, em português e em inglês, e devem ser
% definidas antes do resumo/abstract. Acrescente quantas forem necessárias.
\palavraschave{Grafos, Emparelhamentos, Algoritimos probabilísticos}

\keywords{Graphs, Matchings, Probabilistic algorithms}

% O resumo é obrigatório, em português e inglês. Estes comandos também
% geram automaticamente a referência para o próprio documento, conforme
% as normas sugeridas da USP.
\resumo{
O problema do emparelhamento máximo, que busca encontrar o maior conjunto possível de arestas não adjacentes em um grafo, é um desafio fundamental na teoria dos grafos que tem impulsionado inovações algorítmicas por décadas. 
Em 2009, Nicholas J. A. Harvey alcançou um avanço significativo ao desenvolver um algoritmo probabilístico que resolve o emparelhamento máximo em tempo \(O(n^\omega)\) para grafos arbitrários, onde \(n\) é o número de vértices e \(\omega\) é o expoente da multiplicação de matrizes. 
Seu algoritmo atinge este limite ao combinar de forma engenhosa conceitos da teoria algébrica dos grafos com uma estratégia de implementação eficiente que emprega atualizações "lazy" e técnicas de divisão e conquista. 
Este trabalho fornece uma análise abrangente e implementação do algoritmo de Harvey.
}

\abstract{
The Maximum Matching problem, which seeks to find the largest possible set of non-adjacent edges in a graph, is a fundamental challenge in graph theory that has driven algorithmic innovation for decades. 
In 2009, Nicholas J. A. Harvey made a significant breakthrough by developing a randomized algorithm that solves maximum matching in \(O(n^\omega)\) time for arbitrary graphs, where \(n\) is the number of vertices and \(\omega\) is the matrix multiplication exponent. 
His algorithm achieves this bound by ingeniously combining concepts from algebraic graph theory with an efficient implementation strategy that employs lazy updates and divide-and-conquer techniques. 
This work provides a comprehensive analysis and implementation of Harvey's algorithm.
}

