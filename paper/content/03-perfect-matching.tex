\chapter{Perfect matching}
\label{chap:perfect_matching}

% This chapter explores finding a perfect matching through an algebraic approach, diverging from traditional graph theory methods. The chapter is structured around three key components:

The initial section introduces fundamental algebraic techniques, focusing on Tutte matrices and their probabilistic representation. 
Then, a self-reducible algorithm is presented, demonstrating an algebraic strategy for matching problems. 
The chapter ends with an enhanced version of the self-reducible algorithm that uses rank-two matrix update techniques. 

\section{Tutte Matrix}

\begin{definition}[Indeterminates]
    An \textbf{indeterminate} is a variable that is not assigned a specific value. It represents a symbolic placeholder that never assumes a value.
\end{definition}

\begin{definition}[Tutte Matrix]
\label{def:tutte_matrix}
    Let \(G\) be a graph and \(n \coloneqq |V_G|\).
    For each edge \(uv \in E_G\) associate an indeterminate \(t_{uv}\).
    The Tutte matrix is the \(n \times n\) skew-symmetric matrix such that \(T_{uv} = \pm t_{uv}\) if \(uv \in E_G\) and \(0\), otherwise.
    The sign is chosen such that \(T\) is skew-symmetric.
    The Tutte matrix of \(G\) is denoted as \(T_G\).
\end{definition}

For a Tutte matrix \(T\), its Pfaffian, denoted \(Pf(T)\), is a polynomial whose complete mathematical definition involves sophisticated algebraic concepts beyond our current scope. 
However, two properties of the Pfaffian are crucial:
\begin{enumerate}
    \item The Pfaffian \(Pf(T)\) has a monomial for every perfect matching in the underlying graph of \(T\);
    \item There exists a fundamental relationship between the Pfaffian and the determinant of a Tutte matrix: \(\det(T) = Pf(T)^2\). 
\end{enumerate}

For a comprehensive treatment of this polynomial and its properties, we refer the reader to \citet[Chapter 7]{Godsil:1993}.

\begin{fact}[\citet{Tutte}]
    \label{fact:matching_condition}
    A graph \(G\) has a perfect matching iff \(T_G\) is non-singular.
\end{fact}

\begin{proof}
    Direct from the property \(\det(T) = Pf(T)^2\).
\end{proof}

While this property of Tutte matrices is powerful, it presents a computational challenge for algorithmic applications. 
The issue stems from the relationship \(\det(T) = Pf(T)^2\), and that \(Pf(T)\) contains a monomial for each perfect matching in graph \(G\).
Since a graph may contain exponentially many perfect matchings, symbolic computation becomes infeasible.

\subsection{Probabilistic representation of a Tutte Matrix}
\label{sec:prob_tutte}
\noindent
Fortunately, a probabilistic solution was proposed at \citet{Lovasz:Random}:
\begin{itemize}
    \item Replace the non-zero entries of \(T\) with random values from a sufficiently large field.
\end{itemize}
This provides a computationally feasible approach.

\begin{lemma}[Schwartz-Zippel]
\label{lemma:schwartz-zippel}
Let \(\mathbb{F}\) be a field, and let \(p(x_1, x_2, \dots, x_n)\) be a non-zero polynomial in \(\mathbb{F}[x_1, x_2, \dots, x_n]\) of total degree \(d\). 
Suppose \(S \subseteq \mathbb{F}\) is a finite subset of the field \( \mathbb{F} \). 
If the variables \(x_1, x_2, \dots, x_n\) are chosen independently and uniformly at random from \( S \), then
\[
\Pr\big[p(x_1, x_2, \dots, x_n) = 0\big] \leq \frac{d}{|S|}.
\]
\end{lemma}

\begin{proof}
  See \citet[Theorem 7.2]{MotwaniRaghavan1995}.
\end{proof}

According to \cref{lemma:schwartz-zippel}, selecting a sufficiently large field significantly reduces the probability of failure.
Hence, the rank of a Tutte matrix with random values is preserved with \textbf{high} probability.

\subsubsection{Selecting a sufficiently large field}
Constructing such a field is straightforward. 
For any prime number \(p\), the set of integers modulo \(p\), denoted as \(\Integers_{p}\), forms a field of size \(p\).

\subsubsection{Implementation Overview}

For a graph \(G\). The following assumption is used for the algorithm pertaining Tutte matrices:
\begin{itemize}
    \item For each edge \(uv \in E(G)\), the matrix entry \(t_{uv}\) (since Tutte matrices are skew-symmetric, \(t_{vu}\) is also fixed) is randomly selected from the finite field \(\Integers_p\). 
    These values are not altered throughout the subsequent algorithmic procedures and can be accessed in $O(1)$ time complexity.
%    \item Vertices are mapped to zero-indexed integers, enabling straightforward traversal by iterating from \(0\) to \(n-1\). 
%    This indexing scheme provides a canonical and computationally convenient representation of the graph's structure.
\end{itemize}

\newpage
\section{Naive algorithm}

% A tutte matrix to refer to randomized tutte matrix

Now, a naive algorithm can be implemented.
For a graph \(G \coloneqq (V, E)\).
The idea is, for each edge \(e \in E\), check through \cref{fact:matching_condition} if the graph \((V, E \setminus \{e\})\) has a perfect matching.
\begin{enumerate}
  \item If it does not have a perfect matching, then \(e\) is essential;
  \item Else, let \(G = (V, E \setminus \{e\})\).
\end{enumerate}
Then, after iterating through each edge in \(E\), only the essential edges remain in \(G\), i.e. a perfect matching.
Hence, achieving the following algorithm.

\begin{programruledcaption}{\(\SC{NaiveAlgorithm}\)}
    \label{alg:simple}
    \begin{lstlisting}[
      language={pseudocode},
      style=pseudocode,
      style=wider,
      functions={TutteMatrix, RemoveEdge, RemoveVertex, AddEdge},
      specialidentifiers={},
    ]
function NaiveAlgorithm($G$)
    $T$ := TutteMatrix($G$)
    if $T$ is singular then // $G$ does not have a perfect matching.
      return $\emptyset$
    for \textbf{ each } $uv \in E(G)$ do
        RemoveEdge($T, uv$)
        if $T$ is singular then // By \cref{fact:matching_condition}, edge $uv$ is essential.
            AddEdge($T, uv$)
    return $E(T)$
    \end{lstlisting}
\end{programruledcaption}

\subsubsection{Time complexity}
\noindent
Let \(f(n)\) be the running time of \(\SC{NaiveAlgorithm}\) when \(|V_G| \eqcolon n\).
\begin{itemize}
    \item \textbf{Line 2}: Since a graph has at most \(\binom{n}{2}\), there are at most \(O(n^2)\) iterations;
    \item \textbf{Line 4}: Checking if a matrix is singular can be done in \(O(n^\omega)\) (\cref{matrix:time_complexity}).
\end{itemize}
Thus, the running time can be expressed as:
\[
    f(n) = O(n^2) O(n^\omega) = O(n^{\omega+2}).
\]

\section{Rank-two update algorithm}

The bottleneck of the previous algorithm is the necessity to recompute the whole matrix inverse after each iteration.
An improvement from the previous algorithm was achieved by using \cref{cor:condition_edge_removal} to quickly check essential edges and \cref{thm:rank-two} to update the inverse in \(O(n^2)\) instead of \(O(n^\omega)\).

\begin{theorem}[Rank-two update]
\label{thm:rank-two}
    Let \(G\) be a graph. 
    Let \(T \coloneqq T_G\). 
    Let \(N \coloneqq T^{-1}\).
    Let \(S \subseteq \{u, v\}\) such that \(u, v \in V(G)\) and \(T_{S, S} \neq 0\).
    Let \(\tilde{T}\) be a matrix which is identical to \(T\) except that \(\tilde{T}_{S, S} = 0\).
    Let \(\Delta \coloneqq \tilde{T} - T\).
    If \(\tilde{T}\) is non-singular, then
    \[
        \tilde{T}^{-1} = N + N_{*, S} \cdot 
        \begin{pmatrix}
            1 / (1 + T_{u, v}N_{u, v}) & 0 \\
            0 &  1 / (1 + T_{v, u}N_{v, u})
        \end{pmatrix}
        \cdot \Delta \cdot N_{S, *}.
    \]
\end{theorem}

\begin{proof}
By \cref{cor:update_cor}, it suffices to prove that 
\[
    (I + \Delta N_{S, S})^{-1} = 
    \begin{pmatrix}
        1 / (1 + T_{u, v}N_{u, v}) & 0 \\
        0 &  1 / (1 + T_{v, u}N_{v, u})
    \end{pmatrix}.
\]
Then, one has
\begin{align*}
    (I + \Delta N_{S, S}) &= I + 
    \begin{pmatrix} 0 & -T_{u,v} \\ -T_{v, u} & 0 \end{pmatrix} 
    \begin{pmatrix} 0 & N_{u, v} \\ N_{v, u} & 0\end{pmatrix} & \\
    &= I + 
    \begin{pmatrix} -T_{v, u}N_{u, v} & 0  \\ 0 & -T_{u, v}N_{v, u} \end{pmatrix} & \\
    &= I + 
    \begin{pmatrix} T_{u, v}N_{u, v} & 0  \\ 0 & T_{v, u}N_{v, u} \end{pmatrix} & \text{\(T\) and \(N\) are skew-symmetric} \\
    &= 
    \begin{pmatrix} 1 + T_{u, v}N_{u, v} & 0  \\ 0 & 1 + T_{v, u}N_{v, u} \end{pmatrix}. &
\end{align*}
Finally, 
\begin{align*}
    (I + \Delta N_{S, S})^{-1} 
    &= \begin{pmatrix} 1 + T_{u, v}N_{u, v} & 0  \\ 0 & 1 + T_{v, u}N_{v, u} \end{pmatrix}^{-1} \\
    &= \begin{pmatrix} 1 / (1 + T_{u, v}N_{u, v}) & 0  \\ 0 & 1 / (1 + T_{v, u}N_{v, u}) \end{pmatrix}. \numberthis \label{rank-two:eq}
\end{align*}
Thus, it is proven.
\end{proof}

\begin{corollary}[Edge removal condition]
    \label{cor:condition_edge_removal}
    Let \(G\) be a graph, \(T\) be the Tutte matrix of \(G\) and \(N \coloneqq T^{-1}\).
    An edge \(ij \in E(G)\) is essential if and only if \(N_{i,j} = -1/T_{i,j}\).
\end{corollary}

\begin{proof}
    Direct from \cref{rank-two:eq}. 
\end{proof}

The idea is similar to the naive algorithm.
For a graph \(G = (V, E)\).
For each \(e \in E\), check if \(e\) is essential.
However, instead of temporarily removing \(e\) from \(G\), we use \cref{cor:condition_edge_removal} to quickly decide if \(e\) is essential.
If \(e\) is inessential, then we use a rank-two update instead of completely recomputing the inverse.
Thus, we have the following algorithm.

\begin{programruledcaption}{\(\SC{Rank-two update algorithm}\)}
  \label{alg:rank-two}
    \begin{lstlisting}[
      language={pseudocode},
      style=pseudocode,
      style=wider,
      functions={RankTwoUpdate, TutteMatrix, RemoveEdge, RemoveVertex},
      specialidentifiers={},
    ]
        function RankTwoAlgorithm(G)
            $T$ := TutteMatrix($G$)
            $N$ := $T^{-1}$
            for \textbf{ each } $uv \in E(G)$ do
                if $N_{u,v} \neq -1 / T_{u, v}$ then // By \cref{cor:condition_edge_removal}, this edge is essential
                    S := $\{u, v\}$
                    $N$ := RankTwoUpdate($S, T, N$) // \cref{thm:rank-two}.
            return $E(T)$
    \end{lstlisting}
\end{programruledcaption}

\subsubsection{Time complexity}
\noindent
First, let \(t(n)\) be the running time of \(\SC{Rank2Update}\) when \(|V_G| \eqcolon n\).
Note that 
\[
  \begin{pmatrix} 1 / (1 + T_{u, v}N_{u, v}) & 0  \\ 0 & 1 / (1 + T_{v, u}N_{v, u}) \end{pmatrix}
\]
is a \(2 \times 2\) matrix. 
Consequently,
\[
    t(n) = O(2n^2) = O(n^2).
\]

Then, let \(g(n)\) be the running time of \(\SC{Rank-2 update algorithm}\) when \(n \eqcolon |V_G|\).
Since a Rank-two update is performed for each removed edge and it is necessary to compute the matrix inverse,
we have the following running time.
\begin{align*}
    g(n) = m t(n) + O(n^\omega) = m O(n^2) + O(n^\omega) = O(n^2m) + O(n^\omega).
\end{align*}
