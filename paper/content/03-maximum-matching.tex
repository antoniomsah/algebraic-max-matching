\chapter{Maximum matching}

\section{Perfect Matching}

\begin{definition}[Indeterminates]

\end{definition}


\begin{definition}[Tutte Matrix]
    Given a graph \(G\) and a function that maps every edge of \(G\) to an indeterminate.
    Then, a tutte matrix is a matrix such that, for each \(uv \in E(G)\), one has \(T_{uv} = -T_{vu} = f(vu)\).
\end{definition}

\begin{fact}[Tutte matrix perfect matching condition]
    \label{fact:matching_condition}
    A graph \(G\) has a perfect matching iff \(T_G\) is non-singular.
\end{fact}

\subsection{Basic perfect matching algorithm}

By fact \ref{fact:matching_condition}, one can achieve a very direct algorithm.
The idea is try to remove an edge \(e\), if \(G - e\) has a perfect matching, 
then this edge can be removed; Else, \(e\) belongs to a perfect matching of \(G\).
Repeat this step until only the perfect matching edges are left.
Such approach achieves an \(O(n^{\omega + 2})\) time complexity.

\subsection{Rank-2 update perfect matching algorithm}

One of the bottlenecks of the previous algorithm is the necessity to check after each
edge if the matrix is non-singular, each of these checks is \(O(n^{\omega})\).
Thus, one can maintain the inverse through rank-2 updates.
Achieving a time complexity of \(O(n^{4})\).
For each edge, it suffices to check if \(N_{ij} \neq -1/T_{ij}\).
This condition is direct from corollary \ref{cor:1}.

\subsection{Harvey's perfect matching algorithm}