\chapter{Harvey's algorithm}

TODO: Introduction.

\begin{enumerate}
    \item Brief of the idea;
    \item Pseudo-algorithm;
    \item Corretude;
\end{enumerate}

The main bottleneck in the previous algorithm was the need to update the entire inverse matrix at each step. 
Harvey's algorithm addresses this limitation by employing a divide-and-conquer strategy combined with lazy updates. 
After each recursive step, only the necessary portions of the inverse matrix—those required for the next computation—are updated.
As a result, Harvey's algorithm has a time complexity of \(O(n^\omega)\).

\section{Algorithm}

The algorithm will maintain two matrices: \(T\), a Tutte Matrix of the graph, and \(N\), which is initialized as \(T^{-1}\).
It relies on two recursive functions: \(\SC{DeleteEdgesCrossing}\) and \(\SC{DeleteEdgesWithin}\). 

\begin{programruledcaption}{Harvey's algorithm: DeleteEdgesCrossing}
    \begin{lstlisting}[
      language={pseudocode},
      style=pseudocode,
      style=wider,
      functions={},
      specialidentifiers={},
    ]
        function DeleteEdgesCrossing(R, S)
            if $|R| = 0$ or $|S| = 0$ then return $\emptyset$ // There are no edges

            if $|R| = 1$ and $|S| = 1$ then // There is at most \textbf{one} edge
                r, s := vertex in R, vertex in S
                if $T_{r, s} \neq 0$ \textbf{ and } $T_{r, s} \neq -1 / N_{r, s}$ then // This edge can be removed
                    $N_{r, s}$ := $N_{r, s} (1 - T_{r, s} N_{r, s}) / (1 + T_{r, s} N_{r, s})$ // Update 1
                    $N_{s, r}$ := $-N_{r, s}$
                    $T_{rs}, T_{sr}$ := 0, 0 // Edge has been removed
                end
                return
            end

            $RS$ := $R \cup S$
            $R_1, R_2$ := divide R in two 
            $S_1, S_2$ := divide S in two
            for i \textbf{in} {1, 2} do
                for j \textbf{in} {1, 2} do
                    $T', N'$ := $T, N$
                    $\SC{DeleteEdgesCrossing}(R_i, S_j)$
                    $\Delta$ := $T_{R_i \cup S_j, R_i \cup S_j} - {T'}_{R_i \cup S_j, R_i \cup S_j}$
                    $N_{RS, RS}$ := ${N'}_{RS, RS} - {N'}_{RS, R_i \cup S_j} (I + \Delta {N'}_{R_i \cup S_j, R_i \cup S_j})^{-1} \Delta {N'}_{R_i \cup S_j, RS}$ /
                end
            end
        end
    \end{lstlisting}
\end{programruledcaption}

\begin{programruledcaption}{Harvey's algorithm: DeleteEdgesWithin}
    \begin{lstlisting}[
      language={pseudocode},
      style=pseudocode,
      style=wider,
      functions={},
      specialidentifiers={},
    ]
        function DeleteEdgesWithin(V)
            if |V| = 1 return

            S_1, S_2 := divide S in the middle
            for i \textbf{in} {1, 2} do
                $T', N'$ := $T, N$ // Save current T and N states
                $\SC{DeleteEdgesWithin}(S_i)$
                $\Delta$ := $T_{S_i, S_i} - {T'}_{S_i, S_i}$
                $N_{S, S}$ := $N' - {N'}_{S, S_i}(I + \Delta {N'}_{S, S})^{-1} \Delta {N'}_{S_i, S}$ 
            end
            $\SC{DeleteEdgesCrossing}(S_1, S_2)$
        end
    \end{lstlisting}
\end{programruledcaption}


\begin{programruledcaption}{Harvey's algorithm: Perfect Matching}
    \begin{lstlisting}[
      language={pseudocode},
      style=pseudocode,
      style=wider,
      functions={},
      specialidentifiers={},
    ]
        function PerfectMatching(G)
            T := $\SC{TutteMatrix}(G)$
            if T \textbf{ is singular} then return $\emptyset$ // The graph has no perfect matching
            N := $T^{-1}$
            $\SC{DeleteEdgesWithin(V)}$
            M := $\emptyset$
            for uv \textbf{in} E(G) do
                if $T_{uv} \neq 0$ then M := $M \cup \{uv\}$
            end
            return M
        end
    \end{lstlisting}
\end{programruledcaption}

