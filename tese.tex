% Arquivo LaTeX de exemplo de dissertação/tese a ser apresentada à CPG do IME-USP
%
% Criação: Jesús P. Mena-Chalco
% Revisão: Fabio Kon e Paulo Feofiloff
% Adaptação para UTF8, biblatex e outras melhorias: Nelson Lago
%
% Except where otherwise indicated, these files are distributed under
% the MIT Licence. The example text, which includes the tutorial and
% examples as well as the explanatory comments in the source, are
% available under the Creative Commons Attribution International
% Licence, v4.0 (CC-BY 4.0) - https://creativecommons.org/licenses/by/4.0/


%%%%%%%%%%%%%%%%%%%%%%%%%%%%%%%%%%%%%%%%%%%%%%%%%%%%%%%%%%%%%%%%%%%%%%%%%%%%%%%%
%%%%%%%%%%%%%%%%%%%%%%%%%%%%%%% PREÂMBULO LaTeX %%%%%%%%%%%%%%%%%%%%%%%%%%%%%%%%
%%%%%%%%%%%%%%%%%%%%%%%%%%%%%%%%%%%%%%%%%%%%%%%%%%%%%%%%%%%%%%%%%%%%%%%%%%%%%%%%

% A opção twoside (frente-e-verso) significa que a aparência das páginas pares
% e ímpares pode ser diferente. Por exemplo, as margens podem ser diferentes ou
% os números de página podem aparecer à direita ou à esquerda alternadamente.
% Mas nada impede que você crie um documento "só frente" e, ao imprimir, faça
% a impressão frente-e-verso.
%
% Aqui também definimos a língua padrão do documento (a última da lista) e
% línguas adicionais. Para teses do IME, no mínimo português e inglês são
% obrigatórios, porque independentemente da língua principal do texto é
% preciso fornecer o resumo nessas duas línguas. LaTeX aceita alguns nomes
% diferentes para a língua portuguesa; dentre as opções, prefira sempre
% "brazilian" para português brasileiro e "portuguese" para português europeu.
%\documentclass[a4paper,12pt,twoside,brazilian,english]{book}
\documentclass[a4paper,12pt,twoside,english,brazilian]{book}

% O preâmbulo de um documento LaTeX pode ser razoavelmente longo. Neste
% modelo, optamos por reduzi-lo, colocando praticamente tudo do preâmbulo
% nas packages "imegoodies" e "imelooks".
%
% imegoodies carrega diversas packages muito úteis e populares (algumas
% são praticamente obrigatórias, como amsmath, babel, array etc.). É
% uma boa ideia usá-la com outros documentos também. Ela inclui vários
% comentários explicativos e dicas de uso; não tenha medo de alterá-la
% conforme a necessidade.
%
% imelooks carrega algumas packages e configurações que definem a
% aparência do documento; você também pode querer usá-la (ou partes
% dela) com outros documentos para obter as mesmas fontes, margens
% etc. Tal como "imegoodies", pode valer a pena ler os comentários
% e fazer modificações nessa package. Com a opção "thesis", imelooks
% também define os comandos para capa, folha de rosto etc.
\usepackage{imegoodies}
\usepackage[thesis]{imelooks}

\graphicspath{{figuras/},{fig/},{logos/},{img/},{images/},{imagens/}}

% Comandos rápidos para mudar de língua:
% \en -> muda para o inglês
% \br -> muda para o português
% \texten{blah} -> o texto "blah" é em inglês
% \textbr{blah} -> o texto "blah" é em português
\babeltags{br = brazilian, en = english}


%%%%%%%%%%%%%%%%%%%%%%%%%%%%%%%%%%%%%%%%%%%%%%%%%%%%%%%%%%%%%%%%%%%%%%%%%%%%%%%%
%%%%%%%%%%%%%%%%%%%%%%%%%%%%%%%%%% METADADOS %%%%%%%%%%%%%%%%%%%%%%%%%%%%%%%%%%%
%%%%%%%%%%%%%%%%%%%%%%%%%%%%%%%%%%%%%%%%%%%%%%%%%%%%%%%%%%%%%%%%%%%%%%%%%%%%%%%%

% O arquivo com os dados bibliográficos para biblatex; você pode usar
% este comando mais de uma vez para acrescentar múltiplos arquivos
\addbibresource{bibliografia.bib}

% Este comando permite acrescentar itens à lista de referências sem incluir
% uma referência de fato no texto (pode ser usado em qualquer lugar do texto)
%\nocite{bronevetsky02,schmidt03:MSc, FSF:GNU-GPL, CORBA:spec, MenaChalco08}
% Com este comando, todos os itens do arquivo .bib são incluídos na lista
% de referências
%\nocite{*}

% É possível definir como determinadas palavras podem (ou não) ser
% hifenizadas; no entanto, a hifenização automática geralmente funciona bem
\babelhyphenation{documentclass latexmk soft-ware clsguide} % todas as línguas
\babelhyphenation[brazilian]{Fu-la-no}
\babelhyphenation[english]{what-ever}

% Estes comandos definem o título e autoria do trabalho e devem sempre ser
% definidos, pois além de serem utilizados para criar a capa, também são
% armazenados nos metadados do PDF. O subtítulo é opcional.
\title{Título do trabalho}[um subtítulo]
\translatedtitle{Title of the document}[a subtitle]

\author{Antonio Marcos Shiro Arnauts Hachisuca}

\def\profa{Prof\kern.02em.\kern-.07emª\kern.07em}
\def\dra{Dr\kern-.04em.\kern-.11emª\kern.07em}

% Para TCCs, este comando define o supervisor
\orientador{Prof. Dr. Marcel Kenji de Carli Silva}

% Se não houver, remova; se houver mais de um, basta
% repetir o comando quantas vezes forem necessárias
%\coorientador{Prof. Dr. Ciclano de Tal}
%\coorientador[fem]{\profa{} \dra{} Beltrana de Tal}

\banca{
  \profa{} \dra{} Fulana de Tal (orientadora) -- IME-USP [sem ponto final],
  % Em inglês, não há o "ª"
  %Prof. Dr. Fulana de Tal (advisor) -- IME-USP [sem ponto final],
  Prof. Dr. Ciclano de Tal -- IME-USP [sem ponto final],
  \profa{} \dra{} Convidada de Tal -- IMPA [sem ponto final],
}

% A página de rosto da versão para depósito (ou seja, a versão final
% antes da defesa) deve ser diferente da página de rosto da versão
% definitiva (ou seja, a versão final após a incorporação das sugestões
% da banca).
\tipotese{
  tcc,
  programa={Ciência da Computação},
}

\defesa{
  local={São Paulo},
  data=2024-01-01, % YYYY-MM-DD
}

% A licença do seu trabalho. Use CC-BY, CC-BY-NC, CC-BY-ND, CC-BY-SA,
% CC-BY-NC-SA ou CC-BY-NC-ND para escolher a licença Creative Commons
% correspondente (o sistema insere automaticamente o texto da licença).
% Se quiser estabelecer regras diferentes para o uso de seu trabalho,
% converse com seu orientador e coloque o texto da licença aqui, mas
% observe que apenas TCCs sob alguma licença Creative Commons serão
% acrescentados ao BDTA. Se você tem alguma intenção de publicar o
% trabalho comercialmente no futuro, sugerimos a licença CC-BY-NC-ND.
%
%\direitos{CC-BY-NC-ND}
%
%\direitos{Autorizo a reprodução e divulgação total ou parcial deste
%          trabalho, por qualquer meio convencional ou eletrônico,
%          para fins de estudo e pesquisa, desde que citada a fonte.}
%
%\direitos{I authorize the complete or partial reproduction and disclosure
%          of this work by any conventional or electronic means for study
%          and research purposes, provided that the source is acknowledged.}
%
\direitos{CC-BY}

% Para gerar a ficha catalográfica, acesse https://fc.ime.usp.br/,
% preencha o formulário e escolha a opção "Gerar Código LaTeX".
% Basta copiar e colar o resultado aqui.
\fichacatalografica{}

% Configurações para teoremas, definições, etc.
\usepackage{mdframed}

%%%%%%%%%%%%%% Definitions %%%%%%%%%%%%%%%%%%%%%%
\newcommand{\Naturals}{\mathbb{N}}
\newcommand{\Reals}{\mathbb{R}}

\newcommand{\SC}[1]{\textsc{#1}}

\newmdenv[linecolor=black,linewidth=1pt]{problembox}

\newenvironment{problem}[1]
{
    \begin{problembox}
    \begin{center}
        \Large\SC{#1}
    \end{center}
	\itshape
	\noindent
}
{
    \end{problembox}
}

%%%%%%%%%%%%% Theorems, lemmas, etc. %%%%%%%%%%%%

% Hints
% \newtheoremstyle{maybe} % nome do estilo
% {3pt} % espaço antes
% {3pt} % espaço depois
% {\itshape} % fonte do corpo
% {} % Indentação
% {\bfseries} % fonte do título
% {:} % pontuação após o título
% {.5em} % espaço após o título
% {\thmname{#1}\space(?)\thmnumber{ #2}\thmnote{ (#3)}} % formato do título
% % vazio significa {\thmname{#1}\thmnumber{ #2}\thmnote{ (#3)}}

%% Definition style
\newtheoremstyle{definition} 
{3pt} 
{3pt} 
{} 
{} 
{\bfseries} 
{:} 
{.5em} 
{\thesection.\thmnumber{#2} \thmnote{#3}} 
\theoremstyle{definition}
\newtheorem{definition}{Definition}

\newtheoremstyle{corollary}
{3pt} 
{3pt} 
{\itshape} 
{} 
{\bfseries} 
{:} 
{.5em} 
{\thmname{#1} \thesection.\thmnumber{#2} (\thmnote{#3})} 
\theoremstyle{corollary}
\newtheorem{corollary}{Corollary}

\newtheoremstyle{theorem}
{3pt} 
{3pt} 
{\itshape} 
{} 
{\bfseries} 
{:} 
{.5em} 
{\thmname{#1} \thesection.\thmnumber{#2} (\thmnote{#3})} 
\theoremstyle{theorem}
\newtheorem{theorem}{Theorem}


\begin{document}

%%%%%%%%%%%%%%%%%%%%%%%%%%% CAPA E PÁGINAS INICIAIS %%%%%%%%%%%%%%%%%%%%%%%%%%%%

% Aqui começa o conteúdo inicial que aparece antes do capítulo 1, ou seja,
% página de rosto, resumo, sumário etc. O comando frontmatter faz números
% de página aparecem em algarismos romanos ao invés de arábicos e
% desabilita a contagem de capítulos.
\frontmatter

\pagestyle{plain}

\onehalfspacing % Espaçamento 1,5 na capa e páginas iniciais

\maketitle % capa e folha de rosto

%%%%%%%%%%%%%%%% DEDICATÓRIA, AGRADECIMENTOS, RESUMO/ABSTRACT %%%%%%%%%%%%%%%%%%

\begin{dedicatoria}
Esta seção é opcional e fica numa página separada; ela pode ser usada para
uma dedicatória ou epígrafe.
\end{dedicatoria}

% Reinicia o contador de páginas (a próxima página recebe o número "i") para
% que a página da dedicatória não seja contada.
\pagenumbering{roman}

% Agradecimentos:
% Se o candidato não quer fazer agradecimentos, deve simplesmente eliminar
% esta página. A epígrafe, obviamente, é opcional; é possível colocar
% epígrafes em todos os capítulos. O comando "\chapter*" faz esta seção
% não ser incluída no sumário.
\chapter*{Agradecimentos}
\epigrafe{Do. Or do not. There is no try.}{Mestre Yoda}

Texto texto texto texto texto texto texto texto texto texto texto texto texto
texto texto texto texto texto texto texto texto texto texto texto texto texto
texto texto texto texto texto texto texto texto texto texto texto texto texto
texto texto texto texto. Texto opcional.

%!TeX root=../tese.tex
%("dica" para o editor de texto: este arquivo é parte de um documento maior)
% para saber mais: https://tex.stackexchange.com/q/78101

% As palavras-chave são obrigatórias, em português e em inglês, e devem ser
% definidas antes do resumo/abstract. Acrescente quantas forem necessárias.
\palavraschave{Palavra-chave1, Palavra-chave2, Palavra-chave3}

\keywords{Keyword1,Keyword2,Keyword3}

% O resumo é obrigatório, em português e inglês. Estes comandos também
% geram automaticamente a referência para o próprio documento, conforme
% as normas sugeridas da USP.
\resumo{
Elemento obrigatório, constituído de uma sequência de frases concisas e
objetivas, em forma de texto. Deve apresentar os objetivos, métodos empregados,
resultados e conclusões. O resumo deve ser redigido em parágrafo único, conter
no máximo 500 palavras e ser seguido dos termos representativos do conteúdo do
trabalho (palavras-chave). Deve ser precedido da referência do documento.
Texto texto texto texto texto texto texto texto texto texto texto texto texto
texto texto texto texto texto texto texto texto texto texto texto texto texto
texto texto texto texto texto texto texto texto texto texto texto texto texto
texto texto texto texto texto texto texto texto texto texto texto texto texto
texto texto texto texto texto texto texto texto texto texto texto texto texto
texto texto texto texto texto texto texto texto.
Texto texto texto texto texto texto texto texto texto texto texto texto texto
texto texto texto texto texto texto texto texto texto texto texto texto texto
texto texto texto texto texto texto texto texto texto texto texto texto texto
texto texto texto texto texto texto texto texto texto texto texto texto texto
texto texto.
}

\abstract{
Elemento obrigatório, elaborado com as mesmas características do resumo em
língua portuguesa. De acordo com o Regimento da Pós-Graduação da USP (Artigo
99), deve ser redigido em inglês para fins de divulgação. É uma boa ideia usar
o sítio \url{www.grammarly.com} na preparação de textos em inglês.
Text text text text text text text text text text text text text text text text
text text text text text text text text text text text text text text text text
text text text text text text text text text text text text text text text text
text text text text text text text text text text text text.
Text text text text text text text text text text text text text text text text
text text text text text text text text text text text text text text text text
text text text.
}



%%%%%%%%%%%%%%%%%%%%%%%%%%% LISTAS DE FIGURAS ETC. %%%%%%%%%%%%%%%%%%%%%%%%%%%%%

% Como as listas que se seguem podem não incluir uma quebra de página
% obrigatória, inserimos uma quebra manualmente aqui.
\cleardoublepage

\newcommand\disablenewpage[1]{{\let\clearpage\par\let\cleardoublepage\par #1}}

% Nestas listas, é melhor usar "raggedbottom" (veja basics.tex). Colocamos
% a opção correspondente e as listas dentro de um grupo para ativar
% raggedbottom apenas temporariamente.
\bgroup
\raggedbottom

%%%%% Listas criadas manualmente

%\chapter*{Lista de abreviaturas}
\disablenewpage{\chapter*{Lista de abreviaturas}}

\begin{tabular}{rl}
%  ABNT & Associação Brasileira de Normas Técnicas\\
   URL & Localizador Uniforme de Recursos (\emph{Uniform Resource Locator})\\
   IME & Instituto de Matemática e Estatística\\
   USP & Universidade de São Paulo
\end{tabular}

%\chapter*{Lista de símbolos}
\disablenewpage{\chapter*{Lista de símbolos}}

\begin{tabular}{rl}
%  $\omega$ & Frequência angular\\
%    $\psi$ & Função de análise \emph{wavelet}\\
%    $\Psi$ & Transformada de Fourier de $\psi$\\
\end{tabular}

% Quebra de página manual
\clearpage

%%%%% Listas criadas automaticamente

% Você pode escolher se quer ou não permitir a quebra de página
%\listoffigures
\disablenewpage{\listoffigures}

% Você pode escolher se quer ou não permitir a quebra de página
%\listoftables
\disablenewpage{\listoftables}

\disablenewpage{\listof{program}{\programlistname}}

% Sumário (obrigatório)
\tableofcontents

\egroup % Final de "raggedbottom"

% Referências indiretas ("x", veja "y") para o índice remissivo (opcionais,
% pois o índice é opcional). É comum colocar esses itens no final do documento,
% junto com o comando \printindex, mas em alguns casos isso torna necessário
% executar texindy (ou makeindex) mais de uma vez, então colocar aqui é melhor.
\index{Inglês|see{Língua estrangeira}}
\index{Figuras|see{Floats}}
\index{Tabelas|see{Floats}}
\index{Código-fonte|see{Floats}}
\index{Subcaptions|see{Subfiguras}}
\index{Sublegendas|see{Subfiguras}}
\index{Equações|see{Modo matemático}}
\index{Fórmulas|see{Modo matemático}}
\index{Rodapé, notas|see{Notas de rodapé}}
\index{Captions|see{Legendas}}
\index{Versão original|see{Tese/Dissertação, versões}}
\index{Versão corrigida|see{Tese/Dissertação, versões}}
\index{Palavras estrangeiras|see{Língua estrangeira}}
\index{Floats!Algoritmo|see{Floats, ordem}}


%%%%%%%%%%%%%%%%%%%%%%%%%%%%%%%% CAPÍTULOS %%%%%%%%%%%%%%%%%%%%%%%%%%%%%%%%%%%%%

% Aqui vai o conteúdo principal do trabalho, ou seja, os capítulos que compõem
% a dissertação/tese. O comando mainmatter reinicia a contagem de páginas,
% modifica a numeração para números arábicos e ativa a contagem de capítulos.
\mainmatter

\pagestyle{mainmatter}

% Espaçamento simples
\singlespacing

% A introdução não tem número de capítulo, então os cabeçalhos também não
\pagestyle{unnumberedchapter}
\chapter**{Introdução}
\label{cap:introducao}

\enlargethispage{.5\baselineskip}

Alguma coisa.

\pagestyle{mainmatter}
\chapter{Blossom algorithm}

\enlargethispage{.5\baselineskip}

\section{Matchings}

\begin{definition}[Graph]
	\label{def:graph}
	A \textbf{graph} \(G\) is a triple \((V, E, \varphi)\) such that
	\begin{enumerate}[label=(\roman*)]
		\item \(V\) is the \textbf{vertex set};
		\item \(E\) is the \textbf{edge set};
		\item \(\varphi: E \to V \times V\) is a relation between each edge and a pair of vertices, called the \textbf{incidence function} of \(G\).
	\end{enumerate}
	Usually, it is used 
	\(V(G)\) or \(V_G\) to denote \(V\) and 
	\(E(G)\) or \(E_G\) to denote \(E\).
	Also, if \(e \in E(G)\) and \(\varphi(e) = (u, v)\), then \(u\) and \(v\) are the \textbf{ends} of \(e\);
	When the context is clear, \((u, v)\) may be abbreviated to \(uv\).
\end{definition}

\begin{definition}[Walk]
	\label{def:walk}
	For a graph \(G \coloneqq (V, E, \varphi)\), a \textbf{walk} is a sequence
	\[
		\langle v_0, e_1, v_1, \dots, a_l, v_l \rangle \eqqcolon W
	\]
	such that
	\begin{enumerate}[label=(\roman*)]
		\item \(l \in \Naturals\) is the \textit{length} of \(W\);
		\item \(v_0, v_1, \dots, v_l \in V\);
		\item \(e_1, \dots, e_l \in E\).
	\end{enumerate}
	It is denoted that \(V(W) 
	\coloneqq \{v_0, \dots, v_l\}\) 
	and 
	\(E(W) \coloneqq \{e_1, \dots, e_l\}\). 
	It is said that \(W\) is walk from \(v_0\) to \(v_l\) or a \((v_0, v_l)\)-walk.
	The walk \(W\) is a:
	\begin{itemize}
		\item 
			\textbf{path} if all its vertices are distinct;
		\item 
			\textbf{cycle} if \(v_0 = v_l\);
	\end{itemize}
\end{definition}

\begin{definition}[Matching]
	\label{def:matching}
	For a graph \(G \coloneqq (V, E, \varphi)\), a set \(M \subseteq E\) is a \textbf{matching} of \(G\) if and only if no two edges in \(M\) share an end.
	The matching \(M\) is:
	\begin{itemize}
		\item 
			\textbf{maximal} if there is no edge \(e \in E \setminus M\) such that \(M \cup \{e\}\) is a matching of \(G\);
		\item
			\textbf{maximum} if for every matching \(M'\) of \(G\) one has \(|M| \geq |M'|\).
	\end{itemize}
\end{definition}

Now, the maximum matching problem can be defined as follows:
\\
\begin{problem}{MaxMatching}
	\label{prob:maxmatching}
	Given a graph \(G \coloneqq (V, E, \varphi)\) find a maximum matching of \(G\).
\end{problem}


\section{Maximum matching problem}
Now, the maximum matching problem can be described as:
\\
\begin{problem}{MaxMatching}
	\label{prob:maxmatching}
	Given a graph \(G \coloneqq (V, E, \varphi)\) find a maximum matching of \(G\).
\end{problem}

\begin{definition}[Alternating and augmenting paths]
	Given a matching \(M\) of a graph \(G\).
	A path \(P\) is \(M\)-\textbf{alternating} if its edges are alternating in and out of \(M\). 
	Formally,
	\[
		e_i \in M \Longleftrightarrow e_{i+1} \not \in M \text { for each } i \in [l-1]
		\footnote{For \(n \in \Naturals\), we denote the set \(\{1, \dots, n\}\) as \([n]\).}
	\]
	And \(P\) is \(M\)-\textbf{augmenting} if both \(v_0\) and \(v_l\) are \(M\)-exposed.
\end{definition}

\begin{theorem}[Berge's theorem]
	Let \(G \coloneqq (V, E, \varphi)\) be a graph.
	A matching \(M\) is maximum if and only if there are no \(M\)-augmenting path.
\end{theorem}

\begin{proof}
	Let \(G \coloneqq (V, E, \varphi)\) be a graph and \(M\) be a matching of \(G\).

	\noindent
	\textit{Sufficiency:}
	It will be proven by contradiction.
	Suppose \(M\) is a maximum matching and \(P\) is an \(M\)-augmenting path of \(G\).
	Note that, for \(i \in [l]\), one has: (1) \(e_i \in M\), if \(i\) is even; (2) \(e_i \not \in M\), if \(i\) is odd.

	Let \(M' \coloneqq M \symdiff E(P)\), \(M'\) is a matching since \(v_0, v_l\) are \(M\)-exposed and every vertex in \(\{v_1, \dots, v_{l-1}\}\) is covered by an edge in \(M \cap E(P)\).
	Hence, \(|M'| = |M|+1\), a contradiction. 

	\noindent
	\textit{Necessity:}
	Suppose \(G\) has no \(M\)-augmenting paths.
	Let \(M'\) be a maximum matching of \(G\) and \(G'\) be the graph induced by \(M \symdiff M'\).
	Note that \(G'\) has at least one component and every component of \(G'\) is either a path or a cycle.
	Let \(H\) be a component of \(G'\),
	\begin{enumerate}
		\item 
			if \(|V_H|\) is \textbf{even}, then 
				\(|M \cap V_H| = |M' \cap V_H|\);
		\item 
			if \(|V_H|\) is \textbf{odd}, then \(H\) must be a path. 
			Note that \(M'\) being maximum implies 
			\(|M \cap V_H| > |M' \cap V_H|\) 
			and, consequently,
			\(M' \cap V_H\) is a \(M\)-augmenting path of \(G\).
			Therefore, this case is impossible.
	\end{enumerate}
	Thus, \(|M| = |M'|\), i.e., \(M\) is a maximum matching.
\end{proof}

\begin{definition}[Vertex cover]
	For a graph \(G \coloneqq (V, E, \varphi)\), a subset \(K \subseteq V(G)\) is a \textbf{vertex cover} of \(G\) if every edge of \(E(G)\) has an end in \(K\). 
	A vertex cover is said to be \textbf{minimal} if one has \(|K| \leq |K'|\) for every vertex cover \(K'\) of \(G\).
	Denote \(\tau(G)\) as the size of a minimum vertex cover of \(G\).
\end{definition}

\begin{corollary}[Maximum matching upperbound]
	Let \(G\) be a graph, \(M\) be a matching of \(G\) and \(K\) be a vertex cover of \(G\). 
	Then, \(|M| \leq |K|\)..
\end{corollary}

\begin{proof}
	By definition, every edge of \(M\) has at least one end in \(K\) and no edge in \(M\) share an end.
	Hence, \(|K| \geq |M|\).
\end{proof}

\begin{theorem}[K\"onig's matching theorem]
	Let \(G\) be a bipartite graph, then the maximum size of a matching of \(G\) is equal to the minimum size of a vertex cover of \(G\).
\end{theorem}

\begin{proof}

\end{proof}

Note that K\"onig's theorem \textbf{does not} hold for all graphs;
It suffices to consider a single odd cycle.


\input{conteudo/03-blossom}

%%%%%%%%%%%%%%% SEÇÕES FINAIS (BIBLIOGRAFIA E ÍNDICE REMISSIVO) %%%%%%%%%%%%%%%%

% O comando backmatter desabilita a numeração de capítulos.
\backmatter

\pagestyle{backmatter}

% Espaço adicional no sumário antes das referências / índice remissivo
\addtocontents{toc}{\vspace{2\baselineskip plus .5\baselineskip minus .5\baselineskip}}

% A bibliografia é obrigatória

\printbibliography[
  title=\refname\label{sec:bib}, % "Referências", recomendado pela ABNT
  %title=\bibname\label{sec:bib}, % "Bibliografia"
  heading=bibintoc, % Inclui a bibliografia no sumário
]

\printindex % imprime o índice remissivo no documento (opcional)

\end{document}
